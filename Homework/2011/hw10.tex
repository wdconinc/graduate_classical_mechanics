\documentclass[letterpaper,11pt]{article}
\usepackage[utf8x]{inputenc}
\usepackage{enumerate}
\usepackage{enumitem}
\usepackage{fullpage}
\usepackage{amsmath}
\usepackage{amssymb}
\usepackage{mathrsfs}

\usepackage{pgf}
\usepackage{tikz}
\usepackage{electComp}
\usetikzlibrary{decorations,decorations.pathmorphing,decorations.pathreplacing}

%opening
\title{Physics 601 (Fall 2011) \\ Homework Assignment 10}
\date{Due: Thursday November 10, 2011}

\begin{document}

\maketitle

\paragraph*{Rigid Body Dynamics}
\begin{enumerate}
 \item Show that none of the three principal moments of inertia can exceed the sum of the other two.
 \item Consider a density distribution $\rho(\vec{r}) = \rho(r_\perp,z)$, \textit{i.e.} the $\hat{z}$ axis is an axis of rotational symmetry.
 \begin{enumerate}
  \item Show by direct integration that the $\hat{z}$ axis is a principal axis.
  \item Show that the moments of inertia about any two axis in the $xy$ plane are degenerate.
 \end{enumerate}
 \item What is the ratio of height $h$ to radius $R$ that a uniform circular cylinder should have to have degenerate principal moments ($I_1 = I_2 = I_3$)?  In this case any set of orthonormal axes will be a set of principal axes.
 \item Consider a circular cone of height $h$ and base radius $R$.  Use the parallel axis theorem and the moments of inertia in the frame centered at the vertex of the cone to calculate the principal moments of inertia.
 \item Calculate the principal moments of inertia of a uniform solid sphere with radius $R$.
 \item A uniform solid sphere of mass $M$ and radius $R$ rotates freely in space with an angular velocity $\omega$ about a fixed diameter.  A particle of mass $m$, initially on the pole, moves with a constant velocity along a great circle of the sphere.  Absent any extern torques, show that by the time the particle reaches the other pole the sphere will have been retarded by an angle
 \begin{equation*}
  \alpha = \omega T \left( 1 - \sqrt{\frac{2M}{2M + 5m}} \right)
 \end{equation*}
 where $T$ is the total time required for the particle to move from one pole to the other pole.
\end{enumerate}

\end{document}
