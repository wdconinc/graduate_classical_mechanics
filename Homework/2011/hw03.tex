\documentclass[letterpaper,11pt]{article}
\usepackage[utf8x]{inputenc}
\usepackage{enumerate}
\usepackage{fullpage}
\usepackage{amsmath}

\usepackage{pgf}
\usepackage{tikz}
\usetikzlibrary{arrows,shapes,trees}

%opening
\title{Physics 601 (Fall 2011) \\ Homework Assignment 3}
\date{Due: Thursday September 15, 2011}

\begin{document}

\maketitle

\paragraph*{Generalized Coordinates and the Lagrange's Equations}
\begin{itemize}
 \item Fetter \& Walecka, Problem 3.8.
 \item A system of three masses is arranged on the corners of a foldable frame with length $a$ (\textit{i.e.} the mass $m_2$ can move along the vertical axis as $\theta$ changes).  The whole system is rotating about the vertical axis with angle $\phi$.  Determine the Lagrangian as a function of $\phi$ and $\theta$.  From the equations of motion determine the equilibrium angle for given $\dot\phi = \omega$.
 \begin{center}
  \begin{tikzpicture}
   % Axis
   \draw (0,-3) -- (0,3);
   % Frame
   \draw (0,+2.5) -- node[above right]{$a$} (+2,0) node[right]{$m_1$};
   \draw (0,+2.5) -- node[above left]{$a$} (-2,0) node[left]{$m_1$};
   \draw (+2,0) -- node[below right]{$a$} (0,-2.5) node[below left]{$m_2$};
   \draw (-2,0) -- node[below left]{$a$} (0,-2.5);
   % Angle
   \draw (0,1.5) node[below right]{$\theta$} arc (-90:-51:1);
  \end{tikzpicture}
 \end{center}
\end{itemize}


\paragraph*{Calculus of Variations}
\begin{itemize}
 \item Fetter \& Walecka, Problem 3.10, (a) and (c).
 \item Show that Euler--Lagrange equation for the functional $S = \int_a^b L(q,\dot{q},t) dt$ is equivalent to
  \begin{equation}
   \frac{\partial L}{\partial t} = \frac{d}{dt} \left( L - \dot{q} \frac{\partial L}{\partial \dot{q}} \right).
  \end{equation}
 When $L$ does not depend explicitly on the time $t$, this will lead to a constant of motion.
 \item Fermat's principle of optics states that light will follow the path that leads to an extremum in the travel time $T$.  The velocity of light in a medium with refractive index $n$ is given by $v = c/n$.
 \begin{enumerate}
  \item Derive Snell's law at the interface of two media with refractive indices $n_1$ and $n_2$ by minimizing the functional $T[y(x)]$ for the piecewise linear trajectory $y(x)$.
  \item Light in the Earth's atmosphere: Fetter \& Walecka, Problem 3.12.
 \end{enumerate}
 \item A particle is free to move on the surface of a right circular cone with half vertex angle $\theta_0$.  The position of the particle is given in spherical polar coordinates by the radial distance from the vertex $r$ and the azimuthal angle $\phi$.
 \begin{enumerate}
  \item Show that the geodesics for this surface satisfy the equation
  \begin{equation*}
   r \frac{d^2 r}{d\phi^2} - 2 \left( \frac{dr}{d\phi} \right)^2 = r^2 \sin^2 \theta_0.
  \end{equation*}
  \item Show that the solution to this equation is given by $r = r_0 \sec(\phi - \phi_0) \sin^2 \theta_0$.
 \end{enumerate}
\end{itemize}

\end{document}
