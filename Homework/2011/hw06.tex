\documentclass[letterpaper,11pt]{article}
\usepackage[utf8x]{inputenc}
\usepackage{enumerate}
\usepackage{fullpage}
\usepackage{amsmath}

\usepackage{pgf}
\usepackage{tikz}
\usetikzlibrary{arrows,shapes,trees}

%opening
\title{Physics 601 (Fall 2011) \\ Homework Assignment 6}
\date{Due: Thursday October 6, 2011}

\begin{document}

\maketitle

\paragraph*{Canonical Transformations}
\begin{itemize}
 \item Derive the following direct conditions for canonical transformations:
 \begin{eqnarray*}
  \Bigl( \frac{\partial P_j}{\partial q_i} \Bigr)_{qp} & = & - \Bigl( \frac{\partial p_i}{\partial Q_j} \Bigr)_{QP}, \\
  \Bigl( \frac{\partial P_j}{\partial p_i} \Bigr)_{qp} & = & \Bigl( \frac{\partial q_i}{\partial Q_j} \Bigr)_{QP}, \\
  \Bigl( \frac{\partial Q_j}{\partial q_i} \Bigr)_{qp} & = & \Bigl( \frac{\partial p_i}{\partial P_j} \Bigr)_{QP}, \\
  \Bigl( \frac{\partial Q_j}{\partial p_i} \Bigr)_{qp} & = & - \Bigl( \frac{\partial q_i}{\partial P_j} \Bigr)_{QP}.
 \end{eqnarray*}
 \item Show that the transformation
 \begin{eqnarray*}
  Q = p + i \alpha q, \\
  P = \frac{1}{2i\alpha}(p-i\alpha q),
 \end{eqnarray*}
 is canonical and find a generating function.  Use this transformation to solve the simple linear harmonic osccilator problem.
 \item The Hamiltonian for a system has the form
 \begin{equation*}
  H = \frac{1}{2} \Bigl( \frac{1}{q^2} + p^2 q^4 \Bigr).
 \end{equation*}
 Find the equation of motion for $q$.  Find a canonical transformation that reduces $H$ to the form of a simple harmonic oscillator.  Show that the solution for the transformed variables is such that the equation of motion found for $q$ is satisfied.
\end{itemize}

\paragraph*{Hamilton-Jacobi Theory}
\begin{itemize}
 \item Find Hamilton's principal function $S$ for a point projectile with mass $m$ moving in the vertical $xz$-plane under the constant force of gravity.  From the principal function determine the equations of motion in terms of the constants of motion $\alpha_i$ and $\beta_i$.  Determine the constants of motion for the initial conditions $x=z=0$ and an initial velocity $v_0$ at an angle $\theta$ with the horizontal.
 \item Fetter \& Walecka, Problem 6.11.
\end{itemize}

\end{document}
