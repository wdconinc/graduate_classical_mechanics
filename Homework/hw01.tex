\documentclass[letterpaper,11pt]{article}
\usepackage[utf8x]{inputenc}
\usepackage{enumerate}
\usepackage{enumitem}
\usepackage{fullpage}
\usepackage{amsmath}

\usepackage{pgf}
\usepackage{tikz}
\usetikzlibrary{arrows,shapes,trees}

%opening
\title{Physics 601 (Fall 2013) \\ Homework Assignment 1}
\date{Due: Thursday September 5, 2013}

\begin{document}

\maketitle

\paragraph*{Generalized Coordinates and the Lagrange's Equations}
\begin{enumerate}
 \item Determine the kinetic and potential energy, the Lagrangian, and the equations of motion of a double planar pendulum where both masses are equal to $m$ and both lengths equal to $\ell$.
 \item Two vertical disks of radius $a$ are mounted on the ends of a common axle of length $b$ such that the wheels can rotate independently.  The whole combination rolls without slipping on a horizontal plane.  Show that there are two nonholonomic equations of constraint, $\cos\theta dx + \sin\theta dy = 0$ and $\sin\theta dx - \cos\theta dy = \frac{1}{2} a (d\phi+d\phi^\prime)$, and one holonomic equation of constraint, $\theta = C - \frac{a}{b}(\phi-\phi^\prime)$, with a similar notation as in the previous problem.
 \item A vertical disk of radius $R$ and mass $M$ is rolling without slipping on a horizontal plane.  Determine the constraints in the form $\sum_i^n A_i \dot{q}_i + B = 0$ and determine whether they are holonomic or non-holonomic by explicit calculation.  (Note: you can use $x,y$ for the position of the disk, $\theta$ for the angle of the normal with the $x$ axis, and $\phi$ for the rotation in the plane of the disk.)
 \item Fetter \& Walecka, Chapter 3, Problem 3.3
\end{enumerate}

\end{document}
