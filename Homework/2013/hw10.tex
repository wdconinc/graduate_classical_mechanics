\documentclass[letterpaper,11pt]{article}
\usepackage[utf8x]{inputenc}
\usepackage{enumerate}
\usepackage{enumitem}
\usepackage{hyperref}
\usepackage{fullpage}
\usepackage{amsmath}

\usepackage{pgf}
\usepackage{tikz}
\usetikzlibrary{arrows,shapes,trees}

%opening
\title{Physics 601 (Fall 2013) \\ Homework Assignment 10}
\date{Due: Thursday November 21, 2013}

\begin{document}

\maketitle

\paragraph*{Continuous Systems and Fields}
\begin{enumerate}[resume]
 \item Bead on a string: Fetter \& Walecka, problem 4.17. \emph{Note:} Treat the two halves of the string separately and use boundary and continuity conditions to find the eigenvalue equation for $\omega$.  Obtaining the concise expression in the book requires some trigonometric identities, such as $\tan 2x = 2 \tan x / (1-\tan^2 x)$ and $1 + \tan^2 x = \csc^2 x$.  In part (c) use symmetry properties to rewrite the integral over $(\frac{\ell}{2},\ell)$ as an integral over $(0,\frac{\ell}{2})$.
 \item We derived the Klein-Gordon equation for a charged scalar meson represented by the complex field $\phi$ from the Lagrangian density $\mathcal{L} = c^2 \partial_\mu \phi \partial^\mu \phi^* - m_0^2 c^2 \phi \phi^*$.  Add the term $j_\lambda A^\lambda$ to the Lagrangian density to represent the interaction with an electromagnetic field $A$, where $j_\lambda = i(\phi \partial_\lambda \phi^* - \phi^* \partial_\lambda \phi)$.  What are the field equations for $\phi$ and $\phi^*$?
\end{enumerate}

\paragraph*{Rigid Body Dynamics}
\begin{enumerate}[resume]
 \item Show that none of the three principal moments of inertia can exceed the sum of the other two.
 \item What is the ratio of height $h$ to radius $R$ that a uniform circular cylinder should have to have degenerate principal moments ($I_1 = I_2 = I_3$)?  In this case any set of orthonormal axes will be a set of principal axes.
\end{enumerate}

\end{document}
