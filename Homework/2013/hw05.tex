\documentclass[letterpaper,11pt]{article}
\usepackage[utf8x]{inputenc}
\usepackage{enumerate}
\usepackage{fullpage}
\usepackage{amsmath}

\usepackage{pgf}
\usepackage{tikz}
\usetikzlibrary{arrows,shapes,trees}

%opening
\title{Physics 601 (Fall 2013) \\ Homework Assignment 5}
\date{Due: Thursday October 3, 2013}

\begin{document}

\maketitle

\paragraph*{Hamiltonian Mechanics}
\begin{itemize}
 \item Fetter \& Walecka, Problem 6.2.
 \item Prove Jacobi's identity for Poisson brackets:
 \begin{equation*}
  [A,[B,C]] + [B,[C,A]] + [C,[A,B]] = 0
 \end{equation*}
 Note: You can reduce the amount of tedious algebra considerably by using the following notation: $A_i = \frac{\partial A}{\partial q_i}$ and $A^i = \frac{\partial A}{\partial p_i}$, and by implicitly summing over repated indices: $A_i A^i = \sum_j \frac{\partial A}{\partial q_j} \frac{\partial A}{\partial p_j}$.  This is not related to a similar notation in general relativity.
 \item Fetter \& Walecka, Problem 6.18.
\end{itemize}

\end{document}
