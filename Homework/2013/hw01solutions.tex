\documentclass[letterpaper,11pt]{article}
\usepackage[utf8x]{inputenc}
\usepackage{enumerate}
\usepackage{enumitem}
\usepackage{fullpage}
\usepackage{amsmath}

\usepackage{pgf}
\usepackage{tikz}
\usetikzlibrary{arrows,shapes,trees}

%opening
\title{Physics 601 (Fall 2013) \\ Homework Assignment 1: Solutions}
\date{Due: Thursday September 5, 2013}

\begin{document}

\maketitle

\paragraph*{Generalized Coordinates and the Lagrange's Equations}
\begin{enumerate}
 \item Determine the kinetic and potential energy, the Lagrangian, and the equations of motion of a double planar pendulum where both masses are equal to $m$ and both lengths equal to $\ell$.
\end{enumerate}
Using the positions
\begin{eqnarray*}
 x_1 & = & \ell\sin\theta_1, \\
 y_1 & = & - \ell\cos\theta_1, \\
 x_2 & = & \ell\sin\theta_1 + \ell\sin\theta_2, \\
 y_2 & = & - \ell\cos\theta_1 - \ell\cos\theta_2,
\end{eqnarray*}
we find the potential energy $V = mgy_1 + mgy_2 = -mg(2\ell\cos\theta_1 + \ell\cos\theta_2)$.  Using the velocities
\begin{eqnarray*}
 \dot{x}_1 & = & \ell\dot{\theta}_1\cos\theta_1, \\
 \dot{y}_1 & = & \ell\dot{\theta}_1\sin\theta_1, \\
 \dot{x}_2 & = & \ell\dot{\theta}_1\cos\theta_1 + \ell\dot{\theta}_2\cos\theta_2, \\
 \dot{y}_2 & = & \ell\dot{\theta}_1\sin\theta_1 - \ell\dot{\theta}_2\sin\theta_2,
\end{eqnarray*}
we find the kinetic energy $T = \frac{1}{2}m\left(\dot{x}_1^2 + \dot{y}_1^2 + \dot{x}_2^2 + \dot{y}_2^2\right) = \frac{1}{2}m\ell^2\left(2\dot{\theta}_1^2 + \dot{\theta}_2^2 + 2 \dot{\theta}_1 \dot{\theta}_2 \cos(\theta_1 - \theta_2)\right)$.  The Lagrangian is then
\begin{equation*}
 L = T - V = \frac{1}{2}m\ell^2\left(2\dot{\theta}_1^2 + \dot{\theta}_2^2 + 2 \dot{\theta}_1 \dot{\theta}_2 \cos(\theta_1 - \theta_2)\right) + mg(2\ell\cos\theta_1 + \ell\cos\theta_2).
\end{equation*}
The equations of motion are now
\begin{eqnarray*}
 2m\ell^2\ddot{\theta}_1 + m\ell^2\ddot{\theta}_2\cos(\theta_1 - \theta_2) - m\ell^2\dot{\theta}_2^2\sin(\theta_1 - \theta_2) + 2mg\ell\sin\theta_1 & = & 0, \\
 m\ell^2\ddot{\theta}_2 + m\ell^2\ddot{\theta}_1\cos(\theta_1 - \theta_2) - m\ell^2\dot{\theta}_1^2\sin(\theta_1 - \theta_2) + mg\ell\sin\theta_2 & = & 0.
\end{eqnarray*}

\begin{enumerate}[resume]
 \item A vertical disk of radius $R$ and mass $M$ is rolling without slipping on a horizontal plane.  Determine the constraints in the form $\sum_i^n A_i \dot{q}_i + B = 0$ and determine whether they are holonomic or non-holonomic by explicit calculation.  (Note: you can use $x,y$ for the position of the disk, $\theta$ for the angle of the normal with the $x$ axis, and $\phi$ for the rotation in the plane of the disk.)
\end{enumerate}
The angular velocity of the disk at the contact point should be equal to the linear velocity of the center of the disk.  This means that $a\dot\phi = v$ in magnitude, but also in orientation:
\begin{eqnarray*}
 \dot{x} & = & v \sin\theta = a \dot\phi \sin\theta, \\
 \dot{y} & = & - v \cos\theta = - a \dot\phi \cos\theta.
\end{eqnarray*}
For the first constraint, we get $A_x = 1$, $A_\phi = -a \sin\theta$, and $A_y = A_\theta = B = 0$.  Because $\frac{\partial A_\phi}{\partial \theta} \neq \frac{\partial A_\theta}{\partial \phi}$ this constraint is non-holonomic.  Also the second constraint can be shown to be non-holonomic.

\begin{enumerate}[resume]
 \item Two vertical disks of radius $a$ are mounted on the ends of a common axle of length $b$ such that the wheels can rotate independently.  The whole combination rolls without slipping on a horizontal plane.  Show that there are two non-holonomic equations of constraint, $\cos\theta dx + \sin\theta dy = 0$ and $\sin\theta dx - \cos\theta dy = \frac{1}{2} a (d\phi+d\phi^\prime)$, and one holonomic equation of constraint, $\theta = C - \frac{a}{b}(\phi-\phi^\prime)$, with a similar notation as in the previous problem.
\end{enumerate}
Both disks are constrained similar to the previous problem: $a\dot\phi = v$ and $a\dot\phi' = v'$.  The speed of the center of mass follows then as $\frac{v + v'}{2}$.

The rotation in $\theta$ of the two disks imposes the constraint $v - v' = a\dot\phi - a\dot\phi' = -b\dot\theta$ or $\dot\theta = -\frac{a}{b} (\dot\phi - \dot\phi')$.

The constraint on the components of $v$ are then
\begin{eqnarray*}
 \dot{x} & = & \frac{v + v'}{2} \sin\theta = \frac{a}{2} (\dot\phi + \dot\phi') \sin\theta, \\
 \dot{y} & = & - \frac{v + v'}{2} \cos\theta = - \frac{a}{2} (\dot\phi + \dot\phi') \cos\theta.
\end{eqnarray*}
We can combine these two constraints as
\begin{eqnarray*}
 \sin\theta \dot{x} - \cos\theta \dot{y} & = & \frac{a}{2} (\dot\phi + \dot\phi'), \\
 \cos\theta \dot{x} + \sin\theta \dot{y} & = & 0.
\end{eqnarray*}
If we multiply both of these equations by $dt$ we get the given expression.

We can immediately integrate $\dot\theta = -\frac{a}{b} (\dot\phi - \dot\phi')$ into $\theta = C - \frac{a}{b} (\phi - \phi')$ so it is holonomic.  For the other constraints we have for example $A_x = \sin\theta$, $A_y = -\cos\theta$, $A_\phi = -\frac{a}{2}$, $A_{\phi'} = -\frac{a}{2}$, $A_\theta = B = 0$.  Because $\frac{\partial A_x}{\partial \theta} \neq \frac{\partial A_\theta}{\partial x}$ these constraints are non-holonomic.

\begin{enumerate}[resume]
 \item Fetter \& Walecka, Chapter 3, Problem 3.3
\end{enumerate}
The coordinates of the two masses are
\begin{eqnarray*}
 x_1 & = & x, \\
 x_2 & = & x + \ell\sin\theta, \\
 y_2 & = & -\ell\cos\theta,
\end{eqnarray*}
and the corresponding velocities are
\begin{eqnarray*}
 \dot{x}_1 & = & \dot{x}, \\
 \dot{x}_2 & = & \dot{x} + \ell\dot\theta\cos\theta, \\
 \dot{y}_2 & = & \ell\dot\theta\sin\theta.
\end{eqnarray*}
The Lagrangian is now
\begin{eqnarray*}
 L = T - V & = & \frac{1}{2} m_1 \dot{x}_1^2 + \frac{1}{2} m_2 (\dot{x}_2^2 + \dot{y}_2^2) \\
 & = & \frac{1}{2} m_1 \dot{x}^2 + \frac{1}{2} m_2 (\dot{x}^2 + \ell^2\dot\theta^2 + 2\ell\dot\theta\dot{x}\cos\theta) + m_2 g \ell\cos\theta.
\end{eqnarray*}
The equations of motion are then
\begin{eqnarray*}
 (m_1 + m_2) \ddot{x} + m_2 \ell\ddot\theta\cos\theta - m_2 \ell\dot\theta^2\sin\theta & = & 0, \\
 m_2 \ell^2 \ddot\theta + m_2 \ell\ddot{x} \cos\theta + m_2 g\ell\sin\theta & = & 0.
\end{eqnarray*}
For small oscillations, $\theta \ll 1$, we can approximate this with $\sin\theta \approx \theta$ and $\cos\theta \approx 1$ as
\begin{eqnarray*}
 (m_1 + m_2) \ddot{x} + m_2 \ell\ddot\theta - m_2 \ell\dot\theta^2\theta & = & 0, \\
 m_2 \ell^2 \ddot\theta + m_2 \ell\ddot{x} + m_2 g\ell\theta & = & 0,
\end{eqnarray*}
or
\begin{eqnarray*}
 \ddot{x} & = & - \ell\ddot\theta - g\theta, \\
 m_1 \ell\ddot\theta & = & - (m_1 + m_2) g \theta.
\end{eqnarray*}
This last equation describes harmonic motion with frequency
\begin{equation*}
 \omega^2 = \frac{m_1 + m_2}{m_1} \frac{g}{\ell}.
\end{equation*}


\end{document}
