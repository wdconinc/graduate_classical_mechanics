\documentclass[letterpaper,11pt]{article}
\usepackage[utf8x]{inputenc}
\usepackage{enumerate}
\usepackage{enumitem}
\usepackage{fullpage}
\usepackage{amsmath}

\usepackage{pgf}
\usepackage{tikz}
\usetikzlibrary{arrows,shapes,trees}

%opening
\title{Physics 601 (Fall 2013) \\ Homework Assignment 5: Solutions}
\date{Due: Thursday October 3, 2013}

\begin{document}

\maketitle

\paragraph*{Hamiltonian Mechanics}
\begin{enumerate}
 \item Fetter \& Walecka, Problem 6.2.
\end{enumerate}

Using the Hamiltonian for a free particle in spherical coordinates (previous homework assignment), but with $r$ removed as degree of freedom, the Hamiltonian here is $H = \frac{1}{2 m \ell^2} \left( p_\theta^2 + \frac{p_\phi^2}{\sin^2\theta} \right) - m g \ell \cos\theta$.  This leads to the following Hamilton's equations:
\begin{eqnarray*}
 \dot{\phi} & =  & \frac{1}{m\ell^2} \frac{p_\phi}{\sin^2\theta}, \\
 \dot{p}_\phi & = & 0, \\
 \dot{\theta} & = & \frac{1}{m\ell^2} p_\theta \\
 \dot{p}_\theta & = & \frac{p_\phi^2}{2m\ell^2} \frac{\cos\theta}{\sin^3\theta} - mg\ell\sin\theta.
\end{eqnarray*}
We can eliminate $p_\theta$ from the last two equations and obtain
\begin{equation*}
 m\ell^2\ddot{\theta} = \frac{p_\phi^2}{2m\ell^2} \frac{\cos\theta}{\sin^3\theta} - mg\ell\sin\theta.
\end{equation*}
For uniform circular motion $\dot\phi = \omega$, $\theta = \theta_0$ and $\ddot\theta = \dot\theta = 0$, so
\begin{eqnarray*}
 \omega_0 & = & \frac{1}{m\ell^2} \frac{p_\phi}{\sin^2\theta}, \\
 p_\phi^2 & = & \left(m\ell^2\sin^2\theta_0\right)^2 \frac{g}{\ell\cos\theta_0}.
\end{eqnarray*}
We introduce small deviation from this circular motion, $\theta = \theta_0 + \Delta\theta$ and $\dot\theta = \Delta\dot\theta$, and use
\begin{eqnarray*}
 \frac{1}{\sin^2\theta} & = & \left( \sin\theta_0 \cos\Delta\theta + \cos\theta_0 \sin\Delta\theta \right)^{-2} \\
 & = & \left( \sin\theta_0 \left(1 - \frac{\Delta\theta^2}{2}\right) + \cos\theta_0 \Delta\theta \right)^{-2} + \mathcal{O}(\Delta\theta^3) \\
 & = & \frac{1}{\sin^2\theta_0} \left( 1 + \frac{\cos\theta_0}{\sin\theta_0} \Delta\theta - \frac{\Delta\theta^2}{2} \right)^{-2}  + \mathcal{O}(\Delta\theta^3) \\
 & = & \frac{1}{\sin^2\theta_0} \left( 1 - 2 \left( \frac{\cos\theta_0}{\sin\theta_0} \Delta\theta - \frac{\Delta\theta^2}{2} \right) + 3 \left( \frac{\cos\theta_0}{\sin\theta_0} \Delta\theta \right)^2 \right) + \mathcal{O}(\Delta\theta^3) \\
 & = & \frac{1}{\sin^2\theta_0} \left( 1 - 2 \frac{\cos\theta_0}{\sin\theta_0} \Delta\theta + \left(1 + 3 \frac{\cos^2\theta_0}{\sin^2\theta_0} \right) \Delta\theta^2 \right) + \mathcal{O}(\Delta\theta^3),
\end{eqnarray*}
and
\begin{eqnarray*}
 \cos\theta & = & \cos\theta_0\cos\Delta\theta - \sin\theta_0\sin\Delta\theta \\
 & = & \cos\theta_0 \left( \left(1 - \frac{\Delta\theta^2}{2} \right) - \frac{\sin\theta_0}{\cos\theta_0} \Delta\theta \right) + \mathcal{O}(\Delta\theta^3) \\
 & = & \cos\theta_0 \left( 1 - \frac{\sin\theta_0}{\cos\theta_0} \Delta\theta - \frac{\Delta\theta^2}{2} \right) + \mathcal{O}(\Delta\theta^3).
\end{eqnarray*}
The Hamiltonian becomes now
\begin{eqnarray*}
 H & = & \frac{1}{2m\ell^2} \left( p_{\Delta\theta}^2 + \frac{p_\phi^2}{\sin^2\theta_0} \left( 1 - 2 \frac{\cos\theta_0}{\sin\theta_0} \Delta\theta + \left(1 + 3 \frac{\cos^2\theta_0}{\sin^2\theta_0} \right) \Delta\theta^2 \right) \right) \\
 & & - mg\ell\cos\theta_0 \left( 1 - \frac{\sin\theta_0}{\cos\theta_0} \Delta\theta - \frac{\Delta\theta^2}{2} \right) + \mathcal{O}(\Delta\theta^3) \\
 & = & \frac{1}{2m\ell^2} p_{\Delta\theta}^2 \\
 & & + \frac{1}{2m\ell^2} \left(m\ell^2\sin^2\theta_0\right)^2 \frac{g}{\ell\cos\theta_0} \left( 1 - 2 \frac{\cos\theta_0}{\sin\theta_0} \Delta\theta + \left(1 + 3 \frac{\cos^2\theta_0}{\sin^2\theta_0} \right) \Delta\theta^2 \right) \\
 & & - mg\ell\cos\theta_0 \left( 1 - \frac{\sin\theta_0}{\cos\theta_0} \Delta\theta - \frac{\Delta\theta^2}{2} \right) + \mathcal{O}(\Delta\theta^3) + constant \\
 & = & \frac{1}{2m\ell^2} p_{\Delta\theta}^2 + \frac{1}{2} m\ell^2 \frac{g}{\ell\cos\theta_0} \left( 1 + 3 \cos^2\theta_0 \right) \Delta\theta^2 + \mathcal{O}(\Delta\theta^3) + constant.
\end{eqnarray*}
This is the Hamiltonian for a harmonic oscillator with $\omega^2 = \frac{g}{\ell\cos\theta_0} \left( 1 + 3 \cos^2\theta_0 \right)$.


\begin{enumerate}[resume]
 \item Prove Jacobi's identity for Poisson brackets:
 \begin{equation*}
  [A,[B,C]] + [B,[C,A]] + [C,[A,B]] = 0
 \end{equation*}
 Note: You can reduce the amount of tedious algebra considerably by using the following notation: $A_i = \frac{\partial A}{\partial q_i}$ and $A^i = \frac{\partial A}{\partial p_i}$, and by implicitly summing over repated indices: $A_i A^i = \sum_j \frac{\partial A}{\partial q_j} \frac{\partial A}{\partial p_j}$.  This is not related to a similar notation in general relativity.
\end{enumerate}

With $\partial_i = \frac{\partial}{\partial x_i}$ and $\partial^i = \frac{\partial}{\partial p_i}$ we write
\begin{eqnarray*}
 [A,[B,C]] & = & \sum_i [A,\partial_i B \partial^i C - \partial^i B \partial_i C] \\
 & = & \sum_i [A,\partial_i B \partial^i C] - \sum_i [A,\partial^i B \partial_i C] \\
 & = & \sum_{i,j} \partial_j A \partial_i^j B \partial^i C + \sum_{i,j} \partial_j A \partial_i B \partial^{ij} C - \sum_{i,j} \partial^j A \partial_{ij} B \partial^i C - \sum_{i,j} \partial^j A \partial_i B \partial^i_j C \\
 & & - \sum_{i,j} \partial_j A \partial^{ij} B \partial_i C - \sum_{i,j} \partial_j A \partial^i B \partial_i^j C + \sum_{i,j} \partial^j A \partial^i_j B \partial_i C + \sum_{i,j} \partial^j A \partial^i B \partial_{ij} C.
\end{eqnarray*}
We find similar expressions for $[B,[C,A]]$ and $[C,[A,B]]$.  Inspection shows that all terms will cancel out.

\begin{enumerate}[resume]
 \item Fetter \& Walecka, Problem 6.18.
\end{enumerate}

Using the Levi-Civita symbol $e_{ijk}$, and with $[r_i,p_k] = \delta_{ik}$, we find
\begin{eqnarray*}
 [L_m,L_n] & = & \left[ \sum_{ij} e_{ijm} r_i p_j, \sum_{kl} e_{kln} r_k p_l \right] = \sum_{ij} e_{ijm} \sum_{kl} e_{kln} [r_i p_j, r_k p_l] \\
 & = & \sum_{ij} e_{ijm} \sum_{kl} e_{kln} \left( r_k r_i [p_j, p_l] + r_k [r_i, p_l] p_j + r_i [p_j,r_k] p_l + [r_i, r_k] p_j p_l \right) \\
 & = & \sum_{ij} e_{ijm} \sum_{kl} e_{kln} \left( r_k \delta_{il} p_j - r_i \delta_{jk} p_l \right) \\
 & = & \sum_j \sum_{kl} e_{ljm} e_{kln} r_k p_j - \sum_i \sum_{kl} e_{ikm} e_{kln} r_i p_l \\
 & = & \sum_{jk} (\delta_{jn} \delta_{km} - \delta_{jk} \delta_{nm}) r_k p_j - \sum_{il} (\delta_{lm} \delta_{in} - \delta_{li} \delta_{nm}) r_i p_l \\
 & = & \sum_{ji} (\delta_{jn} \delta_{im} - \delta_{ji} \delta_{nm}) r_i p_j - \sum_{ij} (\delta_{jm} \delta_{in} - \delta_{ji} \delta_{nm}) r_i p_j \\
 & = & \sum_{ij} (\delta_{jn} \delta_{im} - \delta_{jm} \delta_{in}) r_i p_j \\
 & = & \sum_{ij} \sum_{k} e_{mnk} e_{ijk} r_i p_j \\
 & = & \sum_{k} e_{mnk} \sum_{ij} e_{ijk} r_i p_j \\
 & = & \sum_k e_{mnk} L_k,
\end{eqnarray*}
or you could of course just do this for $m = 1$ and $n = 2$ and find $k = 3$ explicitly.  For the magnitude we find
\begin{equation*}
 [L^2,L_n] = [\sum_m L_m L_m,L_n] = 2 \sum_m L_m [L_m,L_n] = 2 \sum_m e_{mnk} L_m L_k = 0.
\end{equation*}
Because the Poisson brackets of the set $L_i$ are different than those of $p_i$ we cannot form a canonical transformation from $p$ to $L$.

\end{document}
