\documentclass[letterpaper,11pt]{article}
\usepackage[utf8x]{inputenc}
\usepackage{enumerate}
\usepackage{enumitem}
\usepackage{fullpage}
\usepackage{amsmath}
\usepackage{amssymb}
\usepackage{mathrsfs}

\usepackage{pgf}
\usepackage{tikz}
\usetikzlibrary{decorations,decorations.pathmorphing,decorations.pathreplacing}

%opening
\title{Physics 601 (Fall 2012) \\ Homework Assignment 11: Solutions}
\date{Due: Friday December 7, 2012}

\begin{document}

\maketitle

\paragraph*{Rigid Body Dynamics}
\begin{enumerate}
 \item A uniform solid sphere of mass $M$ and radius $R$ rotates freely in space with an angular velocity $\omega$ about a fixed diameter.  A particle of mass $m$, initially on the pole, moves with a constant velocity along a great circle of the sphere.  Absent any extern torques, show that by the time the particle reaches the other pole the sphere will have been retarded by an angle
 \begin{equation*}
  \alpha = \omega T \left( 1 - \sqrt{\frac{2M}{2M + 5m}} \right)
 \end{equation*}
 where $T$ is the total time required for the particle to move from one pole to the other pole.
\end{enumerate}
Since there are no external torques, the angular momentum around the $\hat{e}_3$ axis is constant, or
\begin{equation*}
 L = I(t) \vec{\omega}(t) = I_3(t) \omega_3(t) = I_3(0) \omega_3(0).
\end{equation*}
The momentum of inertia around the $\hat{e}_3$ axis is
\begin{equation*}
 I_3(t) = \frac{2}{5}MR^2 + m\left(R\sin\frac{\pi t}{T}\right)^2.
\end{equation*}
The angular velocity is therefore
\begin{equation*}
 \omega_3(t) = \omega \frac{\frac{2}{5}MR^2}{\frac{2}{5}MR^2 + mR^2\sin^2\frac{\pi t}{T}}.
\end{equation*}
The retardation is then
\begin{eqnarray*}
 \alpha & = & \int_0^T \omega dt - \int_0^T \omega_3(t) dt \\
 & = & \omega T - \int_0^T \omega \frac{\frac{2}{5}MR^2}{\frac{2}{5}MR^2 + mR^2\sin^2\frac{\pi t}{T}} dt \\
 & = & \omega T - \int_0^T \omega \frac{1}{1 + \frac{5m}{2M}\sin^2\frac{\pi t}{T}} dt \\
 & = & \omega T \left(1 - \frac{1}{\sqrt{1 + \frac{5m}{2M}}} \right) \\
 & = & \omega T \left(1 - \sqrt{\frac{2M}{2M + 5m}} \right).
\end{eqnarray*}

\begin{enumerate}[resume]
 \item In the case of the torque-free symmetric top, we wrote the Hamiltonian as
 \begin{equation*}
  H = \frac{(p_\alpha - p_\gamma\cos\beta)^2}{2 I_1 \sin^2\beta} + \frac{p^2_\beta}{2 I_1} + \frac{p^2_\gamma}{2 I_3}.
 \end{equation*}
 Because $\alpha$ and $\gamma$ are cyclic, this reduces to a one-dimensional problem in $\beta$.  We can interpret the Hamiltonian as $H = T(p_\beta) + V_{eff}(\beta)$ with
 \begin{eqnarray*}
  T & = & \frac{p^2_\beta}{2 I_1}, \\
  V_{eff} & = & \frac{(p_\alpha - p_\gamma\cos\beta)^2}{2 I_1 \sin^2\beta} + \frac{p^2_\gamma}{2 I_3}.
 \end{eqnarray*}
 Draw schematically how the effective potential energy $V_{eff}$ changes with $\beta$.  What are the two constant-$\beta$ solutions where $V'_{eff}(\beta) = 0$?  Discuss any conditions you might encounter.  What is the resulting total energy corresponding to each solution?
\end{enumerate}
The equilibrium solutions are given by $V'_{eff}(\beta) = 0$, or
\begin{equation*}
 V'_{eff}(\beta) = 2 \frac{(p_\alpha - p_\gamma \cos\beta) p_\gamma \sin\beta}{2 I_1 \sin^2 \beta} - 2 \frac{(p_\alpha - p_\gamma \cos\beta)^2}{2 I_1 \sin^3 \beta} \cos\beta = 0.
\end{equation*}
The solutions for this equation are $\cos\beta_0 = \frac{p_\gamma}{p_\alpha}$.  The resulting total energy is
\begin{eqnarray*}
 E & = & T + V_{eff} = \frac{p_\beta^2}{2 I_1} + \frac{\left(p_\alpha - \frac{p_\gamma^2}{p_\alpha}\right)^2}{2 I_1 \left(1 - \frac{p_\gamma^2}{p_\alpha^2}\right)} + \frac{p_\gamma^2}{2 I_3} \\
 & = & \frac{p_\alpha^2 - p_\gamma^2}{2 I_1} + \frac{p_\beta^2}{2 I_1} + \frac{p_\gamma^2}{2 I_3}.
\end{eqnarray*}
In the first term $p_\alpha^2 - p_\gamma^2$ is equal to $I_1^2 \omega_\perp^2$.


\begin{enumerate}[resume]
 \item Fetter \& Walecka, Problem 5.9.
\end{enumerate}
When we expand the effective potential energy $V_{eff}$ for small $\beta$ we find, with $\cos\beta = 1 - \frac{1}{2}\beta^2 + \frac{1}{24}\beta^4 + \mathcal{O}(\beta^6)$,
\begin{eqnarray*}
 V_{eff}(\beta) & = & \frac{(p_\alpha - p_\gamma\cos\beta)^2}{2 I_1 \sin^2\beta} + \frac{p^2_\gamma}{2 I_3} + Mg\ell \cos\beta \\
 & = & \frac{p_\gamma^2}{2 I_1} \frac{(1 - \cos\beta)^2}{(1 - \cos^2\beta)} + \frac{p^2_\gamma}{2 I_3} + Mg\ell \cos\beta \\
 & = & \frac{p_\gamma^2}{2 I_1} \frac{(1 - \cos\beta)}{(1 + \cos\beta)} + \frac{p^2_\gamma}{2 I_3} + Mg\ell \cos\beta \\
 & = & \frac{p_\gamma^2}{2 I_1} \frac{(\frac{1}{2}\beta^2 - \frac{1}{24}\beta^4)}{(2 - \frac{1}{2}\beta^2)} + \frac{p^2_\gamma}{2 I_3} + Mg\ell \left( 1 - \frac{1}{2}\beta^2 + \frac{1}{24}\beta^4 \right) + \mathcal{O}(\beta^6) \\
 & = & \frac{p_\gamma^2}{2 I_1} \frac{1}{2}\beta^2 (1 - \frac{1}{12}\beta^2) \frac{1}{2} (1 + \frac{1}{4}\beta^2) + \frac{p^2_\gamma}{2 I_3} + Mg\ell \left( 1 - \frac{1}{2}\beta^2 + \frac{1}{24}\beta^4 \right) + \mathcal{O}(\beta^6) \\
 & = & \left(\frac{p^2_\gamma}{2 I_3} + Mg\ell\right) + \frac{1}{2}\beta^2 \left(\frac{p^2_\gamma}{4 I_1} - Mg\ell\right) + \frac{1}{24}\beta^4 \left(\frac{p^2_\gamma}{2 I_1} + Mg\ell\right) + \mathcal{O}(\beta^6).
\end{eqnarray*}
This describes a parabola for small $\beta$, with a minimum at $\beta_0 = 0$ when $p_\gamma^2 > 4 I_1 Mg\ell$.  Explicitly, the derivatives are
\begin{eqnarray*}
 \frac{\partial V_{eff}}{\partial \beta} & = & \beta \left(\frac{p^2_\gamma}{4 I_1} - Mg\ell\right) + \frac{1}{6}\beta^3\left(\frac{p^2_\gamma}{2 I_1} + Mg\ell\right), \\
 \frac{\partial^2 V_{eff}}{\partial \beta^2} & = & \left(\frac{p^2_\gamma}{4 I_1} - Mg\ell\right) + \frac{1}{2}\beta^2\left(\frac{p^2_\gamma}{2 I_1} + Mg\ell\right).
\end{eqnarray*}
For $p_\gamma^2 \lesssim 4 I_1 Mg\ell$ the minimum is at
\begin{eqnarray*}
 \beta_0^2 & = & -3\frac{p_\gamma^2 - 4 I_1 Mg\ell}{p_\gamma^2 + 2 I_1 Mg\ell} \\
 & \approx & -3\frac{p_\gamma^2 - 4 I_1 Mg\ell}{6 I_1 Mg\ell} \\
 & = & 2 - \frac{p_\gamma^2}{2 I_1 Mg\ell}.
\end{eqnarray*}
To find the frequency of oscillations around this equilibrium, we write $\beta = \beta_0 + \eta$ and find
\begin{eqnarray*}
 V_{eff}(\eta) & = & \left(\frac{p^2_\gamma}{2 I_3} + Mg\ell\right) \\
 & & + \frac{1}{2}\beta_0^2 \left(\frac{p^2_\gamma}{4 I_1} - Mg\ell\right) + \frac{1}{24} \beta_0^4 \left(\frac{p^2_\gamma}{2 I_1} + Mg\ell\right) \\
 & & + \eta \beta_0 \left(\frac{p^2_\gamma}{4 I_1} - Mg\ell\right) + \frac{1}{6} \eta \beta_0^3 \left(\frac{p^2_\gamma}{2 I_1} + Mg\ell\right) \\
 & & + \frac{1}{2} \eta^2 \left(\frac{p^2_\gamma}{4 I_1} - Mg\ell\right) + \frac{1}{4} \eta^2 \beta_0^2 \left(\frac{p^2_\gamma}{2 I_1} + Mg\ell\right) \\
 & & + \mathcal{O}(\eta^3) \\
 & = & \hbox{constant} + \frac{1}{2} \left[ \left(\frac{p^2_\gamma}{4 I_1} - Mg\ell\right) + \frac{1}{2} \beta_0^2 \left(\frac{p^2_\gamma}{2 I_1} + Mg\ell\right) \right] \eta^2 + \mathcal{O}(\eta^3),
\end{eqnarray*}
where the first order term is zero explicitly and by definition of the equilibrium.  The frequency is given by
\begin{equation*}
 \Omega^2 = \frac{1}{2 I_1^2} \left( 4 I_1 Mg\ell - p_\gamma^2 \right).
\end{equation*}

\begin{enumerate}[resume]
 \item Fetter \& Walecka, Problem 6.13.
\end{enumerate}
The Hamiltonian can be written as
\begin{equation*}
 H = \frac{(p_\alpha - p_\gamma\cos\beta)^2}{2 I_1 \sin^2\beta} + \frac{p_\beta^2}{2 I_1} + \frac{p_\gamma^2}{2 I_3} + Mg\ell\cos\beta = \frac{p_\beta^2}{2 I_1} + V_{eff}(\beta).
\end{equation*}
Since $\alpha$ and $\gamma$ are cyclic, $p_\alpha$ and $p_\gamma$ are constant.  Therefore we have $S = \alpha p_\alpha + W_\beta + \gamma p_\gamma - E t$.  The Hamilton-Jacobi equation is now
\begin{equation*}
 H\left(\frac{\partial W_\beta}{\partial \beta}\right) = \frac{1}{2 I_1} \left(\frac{\partial W_\beta}{\partial \beta}\right)^2 + V_{eff}(\beta).
\end{equation*}
The characteristic function $W_\beta$ is then
\begin{equation*}
 S = \int \sqrt{2 I_1 \left(E - V_{eff}(\beta)\right)} d\beta - E t,
\end{equation*}
and the final expression is
\begin{equation*}
 \beta + t = \frac{\partial W_\beta}{\partial \beta} = \sqrt{\frac{I_1}{2}} \int \frac{d\beta}{\sqrt{E - V_{eff}(\beta)}}
\end{equation*}



\end{document}
