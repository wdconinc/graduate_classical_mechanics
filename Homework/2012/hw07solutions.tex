\documentclass[letterpaper,11pt]{article}
\usepackage[utf8x]{inputenc}
\usepackage{enumerate}
\usepackage{enumitem}
\usepackage{fullpage}
\usepackage{amsmath}
\usepackage{mathrsfs}

\usepackage{pgf}
\usepackage{tikz}
\usetikzlibrary{arrows,shapes,trees}

%opening
\title{Physics 601 (Fall 2012) \\ Homework Assignment 7: Solutions}
\date{Due: Friday October 26, 2012}

\begin{document}

\maketitle

\paragraph*{Small Oscillations}
\begin{enumerate}
 \item Fetter \& Walecka, Problem 4.3.
\end{enumerate}
a) Introduce the small transverse displacements around the equilibrium $\eta_1 \approx \ell\theta_1$ and $\eta_2 \approx \ell\theta_2$ in the full Lagrangian:
\begin{eqnarray*}
 L & = & T_1 + T_2 - V_1 - V_2 \\
 & = & \frac{1}{2}m_1 \ell^2\dot\theta_1^2 + \frac{1}{2}m_2 (\ell\dot\theta_1 + \ell\dot\theta_2)^2 - m_1g\ell(1 - \cos\theta_1) - m_2g\ell(1 - \cos\theta_1) - m_2g\ell(1 - \cos\theta_2) \\
 & = & \frac{1}{2}m_1 \ell^2\dot\theta_1^2 + \frac{1}{2}m_2 (\ell\dot\theta_1 + \ell\dot\theta_2)^2 - \frac{1}{2}m_1g\ell\theta_1^2 - \frac{1}{2}m_2g\ell\theta_1^2 - \frac{1}{2}m_2g\ell\theta_2^2 \\
 & = & \frac{1}{2}m_1 \dot\eta_1^2 + \frac{1}{2}m_2 (\dot\eta_1 + \dot\eta_2)^2 - \frac{g}{2\ell}m_1\eta_1^2 - \frac{g}{2\ell}m_2\eta_1^2 - \frac{g}{2\ell}m_2\eta_2^2 \\
 & = & \frac{1}{2}m_1 \dot\eta_1^2 + \frac{1}{2}m_2 (\dot\eta_1 + \dot\eta_2)^2 - \frac{g}{2\ell}\left[ (m_1+m_2)\eta_1^2 + m_2\eta_2^2\right].
\end{eqnarray*}
b) The mass and potential matrices are
\begin{eqnarray*}
 M & = & \left[ \begin{array}{cc}
          m_1 + m_2 & m_2 \\
          m_2 & m_2
         \end{array} \right], \\
 V & = & \frac{g}{\ell}
         \left[ \begin{array}{cc}
          m_1 + m_2 & 0 \\
          0 & m_2 
         \end{array} \right].
\end{eqnarray*}
The eigenvalue problem then becomes
\begin{equation*}
 \left| \begin{array}{cc}
  \frac{g}{\ell} (m_1 + m_2) - \omega^2 (m_1 + m_2) & - \omega^2 m_2 \\
  - \omega^2 m_2 & \frac{g}{\ell} m_2 - \omega^2 m_2
 \end{array} \right| = 0.
\end{equation*}
This becomes now the quadratic equation in $\omega^2$
\begin{equation*}
 \left( m_2 (m_1 + m_2) - m_2^2 \right) \omega^4 - 2 \frac{g}{\ell} m_2 (m_1 + m_2) \omega^2 + \frac{g^2}{\ell^2} m_2 (m_1 + m_2) = 0
\end{equation*}
or
\begin{equation*}
 \frac{m_1}{m_1 + m_2} \omega^4 - 2 \frac{g}{\ell} \omega^2  + \frac{g^2}{\ell^2} = 0.
\end{equation*}
With $\gamma^2 = \frac{m_2}{m_1 + m_2}$ we find the positive discriminant
\begin{equation*}
 D = 4 \left(1 - \frac{m_1}{m_1 + m_2}\right) \frac{g^2}{\ell^2} = 4 \frac{m_2}{m_1 + m_2} \frac{g^2}{\ell^2} = 4 \gamma^2 \frac{g^2}{\ell^2}
\end{equation*}
and the eigenvalues are
\begin{eqnarray*}
 \omega_\pm^2 & = & \frac{1}{2} \frac{m_1 + m_2}{m_1} \left( 2 \frac{g}{\ell}  \pm 2 \gamma \frac{g}{\ell} \right) \\
 & = & \frac{g}{\ell} \frac{1 \pm \gamma}{1 - \gamma^2} \\
 & = & \frac{g}{\ell} \frac{1}{1 \mp \gamma}
\end{eqnarray*}
where we used $\frac{m_1}{m_1 + m_2} = 1 - \gamma^2$.

c) To determine the eigenvectors we try to find $z_\pm$ from $(V - \omega_\pm^2 M)z_\pm = 0$, or
\begin{eqnarray*}
 \left[ \begin{array}{cc}
  (m_1 + m_2) \left(\frac{g}{\ell} - \omega_\pm^2\right) & - m_2 \omega_\pm^2 \\
  - m_2 \omega_\pm^2 & m_2 \left(\frac{g}{\ell} - \omega_\pm^2\right)
 \end{array} \right] z_\pm & = & 0 \\
 \left[ \begin{array}{cc}
  (m_1 + m_2) \frac{g}{\ell}\left(1 - \frac{1}{1 \mp \gamma}\right) & - m_2 \frac{g}{\ell} \frac{1}{1 \mp \gamma} \\
  - m_2 \frac{g}{\ell} \frac{1}{1 \mp \gamma} & m_2 \frac{g}{\ell} \left( 1 - \frac{1}{1 \mp \gamma} \right)
 \end{array} \right] z_\pm & = & 0 \\
 m_2 \frac{g}{\ell} \frac{1}{1 \mp \gamma} \left[ \begin{array}{cc}
  \frac{m_1 + m_2}{m_2} \left(1 \mp \gamma - 1\right) & - 1 \\
  - 1 & 1 \mp \gamma - 1
 \end{array} \right] z_\pm & = & 0 \\
 - m_2 \frac{g}{\ell} \frac{1}{1 \mp \gamma} \left[ \begin{array}{cc}
  \pm \frac{1}{\gamma} & 1 \\
  1 & \pm \gamma
 \end{array} \right] z_\pm & = & 0.
\end{eqnarray*}
The eigenvectors are then
\begin{equation*}
 z_\pm = \left[ \begin{array}{c} \mp \gamma \\ 1 \end{array} \right],
\end{equation*}
or, with
\begin{eqnarray*}
 z_\pm^T M z_\pm
   & = & \left[ \begin{array}{cc} \mp \gamma & 1 \end{array} \right]
         \left[ \begin{array}{cc}
          m_1 + m_2 & m_2 \\
          m_2 & m_2
         \end{array} \right]
         \left[ \begin{array}{c} \mp \gamma \\ 1 \end{array} \right] \\
   & = & \left[ \begin{array}{cc} \mp \gamma & 1 \end{array} \right]
         \left[ \begin{array}{c} \mp \gamma (m_1 + m_2) + m_2 \\ \mp \gamma m_2 + m_2 \end{array} \right] \\
   & = & \gamma^2 (m_1 + m_2) \mp \gamma m_2 \mp \gamma m_2 + m_2 \\
   & = & 2 m_2 (1 \mp \gamma),
\end{eqnarray*}
we find the normalized eigenvectors
\begin{equation*}
 \tilde{z}_\pm = \left(2 m_2 (1 \mp \gamma)\right)^{-1/2} \left[ \begin{array}{c} \mp \gamma \\ 1 \end{array} \right],
\end{equation*}

\begin{itemize}
 \item For a very large middle mass, $m_1 \gg m_2$, we have $\gamma = \sqrt{m_2/m_1}$ small.  Both frequencies are $\omega_\pm \approx \frac{g}{\ell}$ and the eigenvectors are
\begin{equation*}
 z_\pm = \left[ \begin{array}{c} \pm m_2/m_1 \\ 1 \end{array} \right] \approx \left[ \begin{array}{c} 0 \\ 1 \end{array} \right].
\end{equation*}
Only the lowest mass is moving with the frequency of a pendulum of length $\ell$.

\item For a very small middle mass, $m_1 \ll m_2$, we have $\gamma^2 \approx 1$.  The eigenvectors are then
\begin{equation*}
 z_\pm = \left[ \begin{array}{c} \pm 1 \\ 1 \end{array} \right].
\end{equation*}
The frequency $\omega_-$ is large and the eigenvector is
\begin{equation*}
 z_- = \left[ \begin{array}{c} -1 \\ 1 \end{array} \right].
\end{equation*}
This corresponds to the middle mass vibrating quickly under the large tension in the string.
The other frequency is $\omega_+ = \frac{g}{2\ell}$ and the corresponding eigenvector is
\begin{equation*}
 z_+ = \left[ \begin{array}{c} 1 \\ 1 \end{array} \right].
\end{equation*}
This describes the motion of a pendulum that is twice as long and with both masses moving in the same direction.
\end{itemize}

d) The modal matrix is
\begin{eqnarray*}
 U & = & \left[ \begin{array}{cc} \tilde{z}_- & \tilde{z}_+ \end{array} \right] \\
   & = & (2 m_2)^{-1/2} \left[ \begin{array}{cc}
          \frac{\gamma}{\sqrt{1 + \gamma}} & - \frac{\gamma}{\sqrt{1 - \gamma}} \\
          \frac{1}{\sqrt{1 + \gamma}} & \frac{1}{\sqrt{1 - \gamma}}
         \end{array} \right].
\end{eqnarray*}
To simplify we use
\begin{eqnarray*}
 \frac{1}{\sqrt{1 \pm \gamma}} & = & \sqrt{1 \mp \gamma} \frac{1}{\sqrt{1 - \gamma^2}} \\
 & = & \sqrt{1 \mp \gamma} \sqrt{\frac{m_1 + m_2}{m_1}} \\
 & = & \sqrt{1 \mp \gamma} \sqrt{\frac{m_2}{m_1}} \frac{1}{\gamma}
\end{eqnarray*}
to find
\begin{eqnarray*}
 U & = & (2 m_1)^{-1/2} \left[ \begin{array}{cc}
          \sqrt{1 - \gamma} & - \sqrt{1 + \gamma} \\
          \gamma^{-1} \sqrt{1 - \gamma} & \gamma^{-1} \sqrt{1 + \gamma}
         \end{array} \right] \\
\end{eqnarray*}
This does indeed diagonalize $M$, by construction, and $V$:
\begin{eqnarray*}
 U^T V U & = & \frac{1}{2 m_1} \frac{g}{\ell}
         \left[ \begin{array}{cc}
          \sqrt{1 - \gamma} & \gamma^{-1} \sqrt{1 - \gamma} \\
          - \sqrt{1 + \gamma} & \gamma^{-1} \sqrt{1 + \gamma}
         \end{array} \right]
         \left[ \begin{array}{cc}
          m_1 + m_2 & 0 \\
          0 & m_2 
         \end{array} \right]
         \left[ \begin{array}{cc}
          \sqrt{1 - \gamma} & - \sqrt{1 + \gamma} \\
          \gamma^{-1} \sqrt{1 - \gamma} & \gamma^{-1} \sqrt{1 + \gamma}
         \end{array} \right] \\
 & = & \frac{1}{2 m_1} \frac{g}{\ell}
         \left[ \begin{array}{cc}
          \sqrt{1 - \gamma} & \gamma^{-1} \sqrt{1 - \gamma} \\
          - \sqrt{1 + \gamma} & \gamma^{-1} \sqrt{1 + \gamma}
         \end{array} \right]
         \left[ \begin{array}{cc}
          (m_1 + m_2) \sqrt{1 - \gamma} & - (m_1 + m_2) \sqrt{1 + \gamma} \\
          m_2 \gamma^{-1} \sqrt{1 - \gamma} & m_2 \gamma^{-1} \sqrt{1 + \gamma}
         \end{array} \right] \\
 & = & \frac{1}{2 m_1} \frac{g}{\ell}
         \left[ \begin{array}{cc}
          (m_1 + m_2) (1 - \gamma) + m_2 \gamma^{-2} (1 - \gamma) & - (m_1 + m_2) \sqrt{1 - \gamma^2} + m_2 \gamma^{-2} \sqrt{1 - \gamma^2} \\
          - (m_1 + m_2) \sqrt{1 - \gamma^2} + m_2 \gamma^{-2} \sqrt{1 - \gamma^2} & (m_1 + m_2) (1 + \gamma) + m_2 \gamma^{-2} (1 + \gamma)
         \end{array} \right] \\
 & = & \frac{1}{2 m_1} \frac{g}{\ell}
         \left[ \begin{array}{cc}
          2 (m_1 + m_2) (1 - \gamma) & 0 \\
          0 & 2 (m_1 + m_2) (1 + \gamma)
         \end{array} \right] \\
 & = & \frac{g}{\ell}
         \left[ \begin{array}{cc}
          \frac{1}{1 + \gamma} & 0 \\
          0 & \frac{1}{1 - \gamma}
         \end{array} \right] = 
         \left[ \begin{array}{cc}
          \omega_-^2 & 0 \\
          0 & \omega_+^2
         \end{array} \right].
\end{eqnarray*}

e) The normal coordinates $\xi_\pm$ are
\begin{eqnarray*}
 \left[ \begin{array}{c}
  \xi_- \\
  \xi_+
 \end{array} \right] & = & U^T M \eta \\
 & = & \frac{1}{\sqrt{2 m_1}}
        \left[ \begin{array}{cc}
         \sqrt{1 - \gamma} & \gamma^{-1} \sqrt{1 - \gamma} \\
         - \sqrt{1 + \gamma} & \gamma^{-1} \sqrt{1 + \gamma}
        \end{array} \right]
        \left[ \begin{array}{cc}
         m_1 + m_2 & m_2 \\
         m_2 & m_2
        \end{array} \right]
        \left[ \begin{array}{c}
         \eta_1 \\
         \eta_2
        \end{array} \right] \\
 & = & \frac{1}{\sqrt{2 m_1}}
        \left[ \begin{array}{cc}
         \sqrt{1 - \gamma} & \gamma^{-1} \sqrt{1 - \gamma} \\
         - \sqrt{1 + \gamma} & \gamma^{-1} \sqrt{1 + \gamma}
        \end{array} \right]
        \left[ \begin{array}{c}
         (m_1 + m_2) \eta_1 + m_2 \eta_2 \\
         m_2 \eta_1 + m_2 \eta_2
        \end{array} \right] \\
 & = & \frac{1}{\sqrt{2 m_1}}
        \left[ \begin{array}{c}
         \sqrt{1 - \gamma} \left(  (m_1 + m_2) \eta_1 + m_2 \eta_2 + \gamma^{-1} (m_2 \eta_1 + m_2 \eta_2)\right) \\
         \sqrt{1 + \gamma} \left(- (m_1 + m_2) \eta_1 - m_2 \eta_2 + \gamma^{-1} (m_2 \eta_1 + m_2 \eta_2)\right)
        \end{array} \right] \\
 & = & \frac{1}{\sqrt{2 m_1}}
        \left[ \begin{array}{c}
         \sqrt{1 - \gamma} \left(  (m_1 + m_2) \eta_1 + \gamma^2 (m_1 + m_2) \eta_2 + \gamma (m_1 + m_2) (\eta_1 + \eta_2)\right) \\
         \sqrt{1 + \gamma} \left(- (m_1 + m_2) \eta_1 - \gamma^2 (m_1 + m_2) \eta_2 + \gamma (m_1 + m_2) (\eta_1 + \eta_2)\right)
        \end{array} \right] \\
 & = & \frac{1}{\sqrt{2 m_1}}
        \left[ \begin{array}{c}
         \sqrt{1 - \gamma} (m_1 + m_2) \left( \eta_1 + \gamma^2 \eta_2 + \gamma (\eta_1 + \eta_2)\right) \\
         \sqrt{1 + \gamma} (m_1 + m_2) \left(-\eta_1 - \gamma^2 \eta_2 + \gamma (\eta_1 + \eta_2)\right)
        \end{array} \right] \\
 & = & \frac{m_1 + m_2}{\sqrt{2 m_1}}
        \left[ \begin{array}{c}
         \sqrt{1 - \gamma} (1 + \gamma) (\eta_1 + \gamma\eta_2) \\
       - \sqrt{1 + \gamma} (1 - \gamma) (\eta_1 - \gamma\eta_2)
        \end{array} \right] \\
 & = & \frac{\sqrt{m_1}}{\sqrt{2} (1 - \gamma^2)}
        \left[ \begin{array}{c}
         \sqrt{1 - \gamma} (1 + \gamma) (\eta_1 + \gamma\eta_2) \\
       - \sqrt{1 + \gamma} (1 - \gamma) (\eta_1 - \gamma\eta_2)
        \end{array} \right] \\
 & = & \frac{\sqrt{m_1}}{\sqrt{2}}
        \left[ \begin{array}{c}
         \frac{1}{\sqrt{1 - \gamma}} (\eta_1 + \gamma\eta_2) \\
       - \frac{1}{\sqrt{1 + \gamma}} (\eta_1 - \gamma\eta_2)
        \end{array} \right].
\end{eqnarray*}
Inversely, the coordinates $\eta$ in terms of the normal coordinates are
\begin{eqnarray*}
 \left[ \begin{array}{c}
  \eta_1 \\
  \eta_2
 \end{array} \right] = U \xi & = & \frac{1}{\sqrt{2 m_1}}
 \left[ \begin{array}{cc}
  \sqrt{1 - \gamma} & - \sqrt{1 + \gamma} \\
  \gamma^{-1} \sqrt{1 - \gamma} & \gamma^{-1} \sqrt{1 + \gamma}
 \end{array} \right]
 \left[ \begin{array}{c}
  \xi_- \\
  \xi_+
 \end{array} \right].
\end{eqnarray*}
 
f) The initial conditions are $\eta_1(0) = a$ and $\eta_2(0) = 0$, and $\dot\eta_1 = \dot\eta_2 = 0$.  The general solution
\begin{equation*}
 \left[ \begin{array}{c}
  \xi_- \\
  \xi_+
 \end{array} \right] = 
 \left[ \begin{array}{c}
  C_- \cos (\omega_- t + \phi_-) \\
  C_+ \cos (\omega_+ t + \phi_+)
 \end{array} \right]
\end{equation*}
gives the initial conditions on the normal coordinates,
\begin{equation*}
 \left[ \begin{array}{c}
  \xi_-(0) \\
  \xi_+(0)
 \end{array} \right] = 
 \left[ \begin{array}{c}
  C_- \cos \phi_- \\
  C_+ \cos \phi_+
 \end{array} \right] = \frac{\sqrt{m_1}}{\sqrt{2}}
 \left[ \begin{array}{c}
  \frac{1}{\sqrt{1 - \gamma}} a \\
- \frac{1}{\sqrt{1 + \gamma}} a
 \end{array} \right],
\end{equation*}
and on their derivatives,
\begin{equation*}
 \left[ \begin{array}{c}
  \dot\xi_-(0) \\
  \dot\xi_+(0)
 \end{array} \right] = 
 \left[ \begin{array}{c}
  - C_- \omega_- \sin \phi_- \\
  - C_+ \omega_+ \sin \phi_+
 \end{array} \right] =
 \left[ \begin{array}{c}
  0 \\
  0
 \end{array} \right],
\end{equation*}
and requires $\phi_- = \phi_+ = 0$.  As already pointed out, for a very large middle mass, $m_1 \gg m_2$, we have $\gamma = \sqrt{m_2/m_1}$ small.  The conditions on $C_-$ and $C_+$ are then
\begin{eqnarray*}
 C_- & \approx & \frac{a\sqrt{m_1}}{\sqrt{2}}, \\
 C_+ & \approx & - \frac{a\sqrt{m_1}}{\sqrt{2}}.
\end{eqnarray*}
Putting all this together, the motion is then given by
\begin{eqnarray*}
 \left[ \begin{array}{c}
  \eta_1 \\
  \eta_2
 \end{array} \right] & \approx &
 \frac{1}{\sqrt{2 m_1}}
 \left[ \begin{array}{cc}
  1 & - 1 \\
  \gamma^{-1} & \gamma^{-1}
 \end{array} \right]
 \left[ \begin{array}{c}
  \xi_- \\
  \xi_+
 \end{array} \right] \\
 & = & 
 \frac{1}{\sqrt{2 m_1}}
 \left[ \begin{array}{c}
  \xi_- - \xi_+ \\
  \gamma^{-1} (\xi_- + \xi_+)
 \end{array} \right] \\
 & = & \frac{1}{\sqrt{2 m_1}} \frac{a\sqrt{m_1}}{\sqrt{2}}
 \left[ \begin{array}{c}
  \cos\omega_-t + \cos\omega_+t \\
  \gamma^{-1} (\cos\omega_-t - \cos\omega_+t)
 \end{array} \right] \\
 & = & 
 a
 \left[ \begin{array}{c}
  \cos\frac{1}{2}(\omega_+ - \omega_-)t \, \cos\frac{1}{2}(\omega_+ + \omega_-)t \\
  \gamma^{-1} \sin\frac{1}{2}(\omega_+ - \omega_-)t \, \sin\frac{1}{2}(\omega_+ + \omega_-)t
 \end{array} \right].
\end{eqnarray*}
Indeed, we find the characteristic beat pattern as discussed in section 23 because $\omega_- \approx \omega_+$, with one of $\eta_1$ and $\eta_2$ oscillating with maximum amplitude when the other one is stationary.

\begin{enumerate}[resume]
 \item Consider two identical harmonic oscillators of mass $m$ and frequency $\omega_0$ that are coupled by the interaction potential $V_{int} = -\alpha q_1 q_2$.
 \begin{itemize}
  \item Determine the normal-mode frequencies and eigenvectors.
  \item Determine the frequencies and discuss their superposition when $\alpha \ll \omega_0^2$.
 \end{itemize}
\end{enumerate}
The Lagrangian for this system is
\begin{equation*}
 L = \frac{1}{2} m \dot{q}_1^2 + \frac{1}{2} m\omega_0^2 q_1^2 + \frac{1}{2} m \dot{q}_2^2 + \frac{1}{2} m\omega_0^2 q_2^2 - \alpha q_1 q_2.
\end{equation*}
The equilibrium occurs when $q_1 = q_2 = 0$.  The mass and potential matrices are
\begin{eqnarray*}
 M & = & m 1, \\
 V & = & m \left[ \begin{array}{cc}
                   \omega_0^2 & -\frac{\alpha}{m} \\
                   -\frac{\alpha}{m} & \omega_0^2
                  \end{array} \right].
\end{eqnarray*}
The eigenvalue equation leads to the frequencies
\begin{equation*}
 \omega_\pm^2 = \omega_0^2 \pm \frac{\alpha}{m}
\end{equation*}
and corresponding eigenvectors
\begin{equation*}
 z_\pm = \frac{1}{\sqrt{2m}} \left[ \begin{array}{cc} \mp 1 \\ 1 \end{array} \right],
\end{equation*}
normalized by setting $z_\pm^T M z_\pm = 1$.

For $\alpha \ll \omega^2$ we get that $\omega_\pm^2 = \omega_0^2 (1 \pm \epsilon)$ and our general solution $q(t) = \sum c_\pm z_\pm \cos\omega_\pm t$ can be written as
\begin{equation*}
 q(t) = \left[ \begin{array}{cc} q_1(t) \\ q_2(t) \end{array} \right] = \frac{c_+ + c_-}{2} \left[ \begin{array}{cc} \cos\frac{\omega_+ - \omega_-}{2}t \cos\frac{\omega_+ + \omega_-}{2}t \\ \sin\frac{\omega_+ - \omega_-}{2}t \sin\frac{\omega_+ + \omega_-}{2}t \end{array} \right].
\end{equation*}
This is a modulation with a small frequency $\frac{\omega_+ - \omega_-}{2} = \omega_0\sqrt{\epsilon}$.

\end{document}
