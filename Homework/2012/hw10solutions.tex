\documentclass[letterpaper,11pt]{article}
\usepackage[utf8x]{inputenc}
\usepackage{enumerate}
\usepackage{enumitem}
\usepackage{fullpage}
\usepackage{amsmath}
\usepackage{amssymb}
\usepackage{mathrsfs}

\usepackage{pgf}
\usepackage{tikz}
\usetikzlibrary{decorations,decorations.pathmorphing,decorations.pathreplacing}

%opening
\title{Physics 601 (Fall 2012) \\ Homework Assignment 10: Solutions}
\date{Due: Friday November 30, 2012}

\begin{document}

\maketitle

\paragraph*{Continuous Systems and Fields}
\begin{enumerate}[resume]
 \item Bead on a string: Fetter \& Walecka, problem 4.17. \emph{Note:} Treat the two halves of the string separately and use boundary and continuity conditions to find the eigenvalue equation for $\omega$.  Obtaining the concise expression in the book requires some trigonometric identities, such as $\tan 2x = 2 \tan x / (1-\tan^2 x)$ and $1 + \tan^2 x = \csc^2 x$.  In part (c) use symmetry properties to rewrite the integral over $(\frac{\ell}{2},\ell)$ as an integral over $(0,\frac{\ell}{2})$.
\end{enumerate}
a) Call $u_1(x,t)$ the displacement of the string between $0$ and $\ell/2$ and $u_2(x,t)$ the displacement between $\ell/2$ and $\ell$.  Of course, $u_1$ and $u_2$ satisfy continuity conditions at $\ell/2$, and boundary conditions at $0$ and $\ell$.  For $x \ne \ell/2$ the string satisfies the wave equation,
\begin{equation*}
 \sigma \frac{\partial^2 u_i}{\partial t^2}\left(x,t\right) = \tau \frac{\partial^2 u_i}{\partial x^2}\left(x,t\right).
\end{equation*}
For $x = \ell/2$ the equation of motion is
\begin{equation*}
 m \frac{\partial^2 u_1}{\partial t^2}\left(\ell/2,t\right) = m \frac{\partial^2 u_2}{\partial t^2}\left(\ell/2,t\right) = \tau \left[ \frac{\partial^2 u_2}{\partial x^2}\left(\ell/2,t\right) - \frac{\partial^2 u_1}{\partial x^2}\left(\ell/2,t\right) \right].
\end{equation*}

To ensure continuity at $\ell/2$ at all times, the time behavior will be the same for $u_1$ and $u_2$.  We can therefore propose a solution of the form
\begin{eqnarray*}
 u_1(x,t) = \rho_1(x) \cos(\omega t + \phi) & \hbox{for} & 0 < x < \ell/2, \\
 u_2(x,t) = \rho_2(x) \cos(\omega t + \phi) & \hbox{for} & \ell/2 < x < \ell.
\end{eqnarray*}
After substitution in the equation of motion for $x \ne \ell/2$, the eigenfunctions $\rho_i(x)$ have to satisfy
\begin{equation*}
 \frac{d^2 \rho_i}{d x^2} + k^2 \rho_i(x) = 0 \quad \hbox{for} \quad 0< x < \ell/2 ~ \hbox{and} ~ \ell/2 < x < \ell.
\end{equation*}
After substitution in the equation of motion for $x = \ell/2$, an additional requirement is
\begin{equation*}
 -m \omega^2 \rho_i\left(\ell/2\right) = \tau \left[ \frac{d^2 \rho_i}{d x^2}\left(\ell/2\right) - \frac{d^2 \rho_i}{d x^2}\left(\ell/2\right) \right] \quad \hbox{for} \quad x = \ell/2.
\end{equation*}

From the equation of motion for $x \ne \ell/2$, the general solution for $\rho_i(x)$ is
\begin{eqnarray*}
 \rho_1(x) & = & A_1 \sin kx + B_1 \cos kx, \\
 \rho_2(x) & = & A_2 \sin kx + B_2 \cos kx.
\end{eqnarray*}
The boundary and continuity conditions on the eigenfunctions $\rho_i(x)$ are
\begin{eqnarray*}
 \rho_1(0) & = & 0, \\
 \rho_1(\ell/2) & = & \rho_2(\ell/2), \\
 \rho_2(\ell) & = & 0.
\end{eqnarray*}
Applying these three conditions to the expressions with four unknowns results in
\begin{eqnarray*}
 \rho_1(x) & = & A_1 \sin kx, \\
 \rho_2(x) & = & A_1 \sin k(\ell - x),
\end{eqnarray*}
with a remaining arbitrary scale factor.  The remaining condition is the equation of motion for the mass $m$, and we find using $\omega = c k$ and $c^2 = \tau / \sigma$
\begin{eqnarray*}
 m \omega^2 A_1 \sin k\ell/2 & = & \tau (A_1 k\cos k\ell/2 + A_1 k\cos k\ell/2) \\
 m \omega^2 & = & 2 \tau k \cot \frac{\omega \ell}{2 c} \\
 \frac{m}{\sigma \ell} & = & \frac{2 c}{\omega \ell} \cot \frac{\omega \ell}{2 c}
\end{eqnarray*}

We could have made this a bit more difficult on ourselves by insisting on the exact form of the general solution, and using the conditions
\begin{eqnarray*}
 \rho_1(0) = 0 & \Rightarrow & B_1 = 0, \\
 \rho_2(\ell) = 0 & \Rightarrow & B_2 = -A_2 \tan k\ell, \\
 \rho_1(\ell/2) = \rho_2(\ell/2) & \Rightarrow & A_1 = A_2 (1 - \cot k\ell/2 \tan k\ell).
\end{eqnarray*}
Plugging all this into the equation of motion for the mass $m$ gives
\begin{eqnarray*}
 m \omega^2 A_1 \sin k\ell/2 & = & \tau \left(A_1 k \cos k\ell/2 - B_1 k \sin k\ell/2 - A_2 k \cos k\ell/2 + B_2 k \sin k\ell/2 \right) \\
 m \omega^2 A_1 & = & \tau k \left( A_1 \cot k\ell/2 - A_2 \cot k\ell/2 + B_2 \right) \\
 m \omega^2 A_2 (1 - \cot k\ell/2 \tan k\ell) & = & A_2 \tau k \left( \left(1 - \cot k\ell/2 \tan k\ell\right) \cot k\ell/2 - \cot k\ell/2 - \tan k\ell \right).
\end{eqnarray*}
Using $\tan 2 a = 2 \tan a / (1 - \tan^2 a)$ and $1 + \tan^2 a = 1 / \cos^2 a$ we can simplify further
\begin{eqnarray*}
 \frac{m \omega}{c \sigma} (1 - \cot k\ell/2 \tan k\ell) & = & - \tan k\ell ( 1 + \cot^2 k\ell/2) \\
 & = & \frac{- \tan k\ell ( 1 + \cot^2 k\ell/2)}{1 - \cot k\ell/2 \tan k\ell} \\
 & = & 2 \frac{\tan k\ell/2 + \cot k\ell/2)}{1 + \tan^2 k\ell/2} \\
 & = & 2 \cos^2 k\ell/2 (\tan k\ell/2 + \cot k\ell/2) \\
 & = & 2 \cot k\ell/2 = 2 \cot \frac{\omega \ell}{2 c}.
\end{eqnarray*}
This finally gives us the expression found earlier:
\begin{equation*}
 \frac{2 c}{\omega \ell} \cot \frac{\omega \ell}{2 c} = \frac{m}{\sigma \ell}.
\end{equation*}

For the limit $m \to 0$, the relation reduces to
\begin{eqnarray*}
 \frac{2 c}{\omega \ell} \cot \frac{\omega \ell}{2 c} & = & 0, \\
 \frac{\omega \ell}{2 c} = \frac{k\ell}{2} & = & (2 n + 1) \frac{\pi}{2}, \\
 k & = & (2 n + 1) \frac{\pi}{\ell}.
\end{eqnarray*}
These are the odd wavenumbers for the motion of a simple string with fixed ends, $k = \frac{n \pi}{\ell}$ with $n = 1, 2,\ldots$  Only for the odd modes does the mass move at all.  The even modes keep the central mass stationary at all times.

For the limit $m \to \infty$, the relation becomes
\begin{eqnarray*}
 \frac{2 c}{\omega \ell} \cot \frac{\omega \ell}{2 c} & \to & \infty, \\
 \frac{\omega \ell}{2 c} = \frac{k\ell}{2} & = & n \pi, \\
 k & = & \frac{n \pi}{\ell/2}.
\end{eqnarray*}
These are the wavenumbers for the motion of a simple string of length $\ell/2$ with fixed ends.  The central point with infinite mass acts as a fixed point.

c) The integral can be evaluated as
\begin{eqnarray*}
 \int_0^\ell dx \rho_p(x) m(x) \rho_q(x) & = & \int_0^{\ell/2} dx \rho_p(x) \sigma \rho_q(x) + m \rho_p(\ell/2) \rho_q(\ell/2) + \int_{\ell/2}^\ell dx \rho_p(x) \sigma \rho_q(x) \\
 & = & 2 \sigma \int_0^{\ell/2} dx \sin k_p x \sin k_q x + m \sin k_p \ell/2 \sin k_q \ell/2,
\end{eqnarray*}
where we used symmetry properties.

If $k_p \ne k_q$ we can evaluate the first integral as
\begin{eqnarray*}
 \int_0^{\ell/2} dx \sin k_p x \sin k_q x & = & \frac{1}{2} \int_0^{\ell/2} dx \left[ \cos (k_p - k_q) x - \cos (k_q + k_p) x \right] \\
 & = & \frac{1}{2} \frac{1}{k_p - k_q} \sin (k_p - k_q) \ell/2 - \frac{1}{2} \frac{1}{k_p + k_q} \sin (k_q + k_p) \ell/2 \\
 & = & \frac{1}{2} \frac{1}{k_p - k_q} \left( \sin k_p\ell/2 \cos k_q \ell/2 - \cos k_p\ell/2 \sin k_q \ell/2 \right) \\
 & & - \frac{1}{2} \frac{1}{k_p + k_q} \left( \sin k_q \ell/2 \cos k_p \ell/2 + \cos k_q \ell/2 \sin k_p \ell/2 \right) \\
 & = & \frac{1}{2} \frac{1}{k_p^2 - k_q^2} \left( (k_p + k_q) \sin k_p\ell/2 \cos k_q \ell/2 - (k_p + k_q) \cos k_p\ell/2 \sin k_q \ell/2 \right) \\
 & & - \frac{1}{2} \frac{1}{k_p^2 - k_q^2} \left( (k_p - k_q) \sin k_q \ell/2 \cos k_p \ell/2 + (k_p - k_q) \cos k_q \ell/2 \sin k_p \ell/2 \right) \\
 & = & \frac{1}{k_p^2 - k_q^2} \left( k_q \cos k_q \ell/2 \sin k_p \ell/2 - k_p \cos k_p \ell/2 \sin k_q \ell/2 \right).
\end{eqnarray*}
In the last expression we can use the result of part b to rewrite $\cos k\ell/2 = \frac{m}{\sigma\ell} k\ell/2 \sin k\ell/2$.
\begin{eqnarray*}
 2 \sigma \int_0^{\ell/2} dx \sin k_p x \sin k_q x & = & 2 \sigma \frac{1}{k_p^2 - k_q^2} \frac{m}{\sigma \ell} \frac{\ell}{2} (k_q^2 - k_p^2) \sin k_p \ell/2 \sin k_q \ell/2 \\
 & = & -m \sin k_p \ell/2 \sin k_q \ell/2.
\end{eqnarray*}
The total integral is therefore zero.

If $k_p = k_q$ we can evaluate the first integral as
\begin{eqnarray*}
 \int_0^{\ell/2} dx \sin^2 k_p x & = & \frac{1}{2} \int_0^{\ell/2} dx ( 1 - \cos 2 k_p x) \\
 & = & \frac{1}{2} \left( \frac{\ell}{2} - \frac{1}{2 k_p} \sin 2 k_p \ell/2 \right) \\
 & = & \frac{1}{4} \left( \ell - \frac{1}{k_p} \sin k_p \ell \right) \\
 & = & \frac{1}{4} \left( \ell - \frac{2}{k_p} \sin k_p \ell/2 \cos k_p \ell/2 \right) \\
 & = & \frac{1}{4} \left( \ell - \frac{m}{\sigma} \sin^2 k_p \ell/2 \right).
\end{eqnarray*}
The total integral is now
\begin{equation*}
 \int_0^\ell dx \rho_p(x) m(x) \rho_q(x) = \frac{\sigma \ell}{2} + \frac{m}{2} \sin^2 k_p \ell/2,
\end{equation*}
and this can be used to properly normalize the eigenfunctions.

\begin{enumerate}[resume]
 \item We derived the Klein-Gordon equation for a charged scalar meson represented by the complex field $\phi$ from the Lagrangian density $\mathcal{L} = c^2 \partial_\mu \phi \partial^\mu \phi^* - m_0^2 c^2 \phi \phi^*$.  Add the term $j_\lambda A^\lambda$ to the Lagrangian density to represent the interaction with an electromagnetic field $A$, where $j_\lambda = i(\phi \partial_\lambda \phi^* - \phi^* \partial_\lambda \phi)$.  What are the field equations for $\phi$ and $\phi^*$?
\end{enumerate}
The Lagrangian density with the added interaction term is
\begin{eqnarray*}
 \mathcal{L} & = & c^2 \partial_\mu \phi \partial^\mu \phi^* - m_0^2 c^2 \phi \phi^* + j_\lambda A^\lambda \\
 & = & c^2 \partial_\mu \phi \partial^\mu \phi^* - m_0^2 c^2 \phi \phi^* + i(\phi \partial_\lambda \phi^* - \phi^* \partial_\lambda \phi) A^\lambda.
\end{eqnarray*}
Using
\begin{eqnarray*}
 \frac{\partial \mathcal{L}}{\partial(\partial_\mu\phi)} & = & c^2 \partial^\mu \phi^* + i A^\mu \phi^*, \\
 \frac{\partial \mathcal{L}}{\partial \phi} & = & - m_0^2 c^2 \phi^*,
\end{eqnarray*}
the field equations are then
\begin{equation*}
 c^2 \partial_\mu \partial^\mu \phi - i (\partial_\mu A^\mu) \phi - i (\partial_\mu \phi) A^\mu + m_0^2 c^2 \phi = 0,
\end{equation*}
and similar for $\phi^*$.

\paragraph*{Rigid Body Dynamics}
\begin{enumerate}
 \item Show that none of the three principal moments of inertia can exceed the sum of the other two.
\end{enumerate}
Since we are talking about the principal moments, our coordinate axes are the principal axes and only $I_{ii} = I_i \ne 0$.  The principal moment around $\hat{e}_1$ is
\begin{eqnarray*}
 I_1 & = & \int d^3\vec{r} \rho(\vec{r}) (x_2^2 + x_3^2) \\
 & = & \int d^3\vec{r} \rho(\vec{r}) (x_1^2 + x_2^2) + \int d^3\vec{r} \rho(\vec{r}) (x_1^2 + x_3^2) - 2 \int d^3\vec{r} \rho(\vec{r}) x_1^2 \\
 & = & I_2 + I_3 - 2 \int d^3\vec{r} \rho(\vec{r}) x_1^2.
\end{eqnarray*}
Because the integrand is always positive, this means that $I_1 \le I_2 + I_3$.  The same argument can be made for the two other principal moments.

\begin{enumerate}[resume]
 \item Consider a density distribution $\rho(\vec{r}) = \rho(r_\perp,z)$, \textit{i.e.} the $\hat{z}$ axis is an axis of rotational symmetry.
 \begin{enumerate}
  \item Show by direct integration that the $\hat{z}$ axis is a principal axis.
  \item Show that the moments of inertia about any two axis in the $xy$ plane are degenerate.
 \end{enumerate}
\end{enumerate}
a) The $\hat{z}$ axis will be a principal axis if $I_{13}$ and $I_{23}$ are zero.
\begin{eqnarray*}
 I_{13} & = & - \int d^3\vec{r} \rho(\vec{r}) x z \\
 & = & - \int d^3\vec{r} \rho(r_\perp,z) r_\perp \cos\phi z \\
 & = & - \int_0^{+\infty} dr_\perp \int_0^{2\pi} r_\perp d\phi \int_{-\infty}^{+\infty} dz \rho(r_\perp,z) r_\perp \cos\phi z \\
 & = & - \int_0^{+\infty} dr_\perp r_\perp^2 \int_{-\infty}^{+\infty} dz z \rho(r_\perp,z) \int_0^{2\pi} d\phi \cos\phi \\
 & = & 0,
\end{eqnarray*}
due to the $\cos\phi$ integral.  Similarly, $I_{23} = 0$, and therefore $I_{33} = I_3$ is a principal moment around the principal axis $\hat{z}$, with
\begin{eqnarray*}
 I_{33} & = & \int d^3\vec{r} \rho(r_\perp,z) r_\perp^2 \\
 & = & \int_0^{+\infty} dr_\perp \int_0^{2\pi} r_\perp d\phi \int_{-\infty}^{+\infty} dz \rho(r_\perp,z) r_\perp^2 \\
 & = & 2 \pi \int_0^{+\infty} r_\perp^3 dr_\perp \int_{-\infty}^{+\infty} dz \rho(r_\perp,z).
\end{eqnarray*}

b) The intertia tensor can be written as
\begin{equation*}
 I = \left[ \begin{array}{ccc}
  I_{11} & I_{12} & 0 \\
  I_{12} & I_{22} & 0 \\
  0 & 0 & I_{33}
 \end{array} \right].
\end{equation*}
We can determine the remaining components of this tensor (with $r = r_\perp$):
\begin{eqnarray*}
 I_{12} & = & - \int d^3\vec{r} \rho(r_\perp,z) r_\perp \cos\phi r_\perp \sin\phi \\
 & = & - \int_0^{+\infty} r_\perp^3 dr_\perp \int_{-\infty}^{+\infty} dz \rho(r_\perp,z) \int_0^{2\pi} d\phi \cos\phi \sin\phi \\
 & = & - \int_0^{+\infty} r_\perp^3 dr_\perp \int_{-\infty}^{+\infty} dz \rho(r_\perp,z) \int_0^{2\pi} d\phi \sin\phi d \sin\phi \\
 & = & 0 \\
 I_{11} & = & \int d^3\vec{r} \rho(r_\perp,z) (r_\perp^2 \cos^2\phi + z^2) \\
 I_{22} & = & \int d^3\vec{r} \rho(r_\perp,z) (r_\perp^2 \sin^2\phi + z^2) \\
 & = & \int_0^{+\infty} dr_\perp \int_{-\infty}^{+\infty} dz \rho(r_\perp,z) \int_0^{2\pi} d\phi (r_\perp^2 \sin^2\phi + z^2) \\
 & = & \int_0^{+\infty} dr_\perp \int_{-\infty}^{+\infty} dz \rho(r_\perp,z) \left( r_\perp^3 \int_0^{2\pi} d\phi \sin^2\phi + 2 \pi r_\perp z^2 \right) \\
 & = & \int_0^{+\infty} dr_\perp \int_{-\infty}^{+\infty} dz \rho(r_\perp,z) \left( \pi r_\perp^3 + 2 \pi r_\perp z^2 \right) \\
 & = & \int_0^{+\infty} dr_\perp \int_{-\infty}^{+\infty} dz \rho(r_\perp,z) \left( r_\perp^3 \int_0^{2\pi} d\phi \cos^2\phi + 2 \pi r_\perp z^2 \right) \\
 & = & I_{11}.
\end{eqnarray*}
The inertia tensor can therefore be written as
\begin{equation*}
 I = \left[ \begin{array}{ccc}
  I_{11} & 0 & 0 \\
  0 & I_{11} & 0 \\
  0 & 0 & I_{33}
 \end{array} \right].
\end{equation*}
The moments of inertia about any (degenerate) principal axes in the $xy$ plane are equal.

\begin{enumerate}[resume]
 \item What is the ratio of height $h$ to radius $R$ that a uniform circular cylinder should have to have degenerate principal moments ($I_1 = I_2 = I_3$)?  In this case any set of orthonormal axes will be a set of principal axes.
\end{enumerate}
Because of the previous problem we know that $I_1 = I_2$, and for this circular cylinder
\begin{eqnarray*}
 I_1 = I_2 & = & \int_0^R dr \int_{-h/2}^{+h/2} dz \rho \left( \pi r^3 + 2 \pi r z^2 \right) \\
 & = & \frac{1}{12} M (3 R^2 + h^2).
\end{eqnarray*}
The third principal moment is
\begin{eqnarray*}
 I_3 & = & 2 \pi \rho \int_0^R r^3 dr \int_{-h/2}^{+h/2} dz \\
 & = & \frac{1}{2} M R^2.
\end{eqnarray*}
The principal moments will be equal if $6 R^2 = 3 R^2 + h^2$ or $h = \sqrt{3} R$.

\begin{enumerate}[resume]
 \item Consider a circular cone of height $h$ and base radius $R$.  Use the parallel axis theorem and the moments of inertia in the frame centered at the vertex of the cone to calculate the principal moments of inertia.
\end{enumerate}
The mass of the cone is
\begin{equation*}
 M = 2 \pi \rho \int_0^h dz \int_0^{R \frac{z}{h}} dr r = \frac{1}{3} \rho \pi R^2 h.
\end{equation*}
The center of mass is at a height $a$ given by
\begin{equation*}
 M a = 2 \pi \rho \int_0^h dz \int_0^{R \frac{z}{h}} dr r z = \frac{1}{4} \rho \pi R^2 h^2,
\end{equation*}
or $a = \frac{3}{4} h$.
Centered at the vertex of the cone we have
\begin{eqnarray*}
 I_{11} = I_{22} & = & \int_0^h dz \int_0^{R \frac{z}{h}} dr \rho \left( \pi r^3 + 2 \pi r z^2 \right) \\
 & = & \frac{3}{20} M (R^2 + 4 h^2), \\
 I_{33} & = & 2 \pi \rho \int_0^h dz \int_0^{R \frac{z}{h}} dr r^3 \\
 & = & \frac{3}{10} M R^2.
\end{eqnarray*}
We can use the parallel axis theorem to relate this to the moments of inertia around the center of mass,
\begin{eqnarray*}
 \bar{I}_1 = \bar{I}_2 & = & I_{11} - M \left(\frac{3}{4} h\right)^2 \\
 & = & \frac{3}{20} M \left(R^2 + \frac{1}{4} h^2 \right), \\
 \bar{I}_3 & = & I_{33} = \frac{3}{10} M R^2.
\end{eqnarray*}

\end{document}
