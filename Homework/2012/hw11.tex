\documentclass[letterpaper,11pt]{article}
\usepackage[utf8x]{inputenc}
\usepackage{enumerate}
\usepackage{enumitem}
\usepackage{fullpage}
\usepackage{amsmath}
\usepackage{amssymb}
\usepackage{mathrsfs}

\usepackage{pgf}
\usepackage{tikz}
\usetikzlibrary{decorations,decorations.pathmorphing,decorations.pathreplacing}

%opening
\title{Physics 601 (Fall 2012) \\ Homework Assignment 11}
\date{Due: Friday December 7, 2012}

\begin{document}

\maketitle

\paragraph*{Rigid Body Dynamics}
\begin{enumerate}
 \item A uniform solid sphere of mass $M$ and radius $R$ rotates freely in space with an angular velocity $\omega$ about a fixed diameter.  A particle of mass $m$, initially on the pole, moves with a constant velocity along a great circle of the sphere.  Absent any extern torques, show that by the time the particle reaches the other pole the sphere will have been retarded by an angle
 \begin{equation*}
  \alpha = \omega T \left( 1 - \sqrt{\frac{2M}{2M + 5m}} \right)
 \end{equation*}
 where $T$ is the total time required for the particle to move from one pole to the other pole.
 \item In the case of the torque-free symmetric top, we wrote the Hamiltonian as
 \begin{equation*}
  H = \frac{(p_\alpha - p_\gamma\cos\beta)^2}{2 I_1 \sin^2\beta} + \frac{p^2_\beta}{2 I_1} + \frac{p^2_\gamma}{2 I_3}.
 \end{equation*}
 Because $\alpha$ and $\gamma$ are cyclic, this reduces to a one-dimensional problem in $\beta$.  We can interpret the Hamiltonian as $H = T(p_\beta) + V_{eff}(\beta)$ with
 \begin{eqnarray*}
  T & = & \frac{p^2_\beta}{2 I_1}, \\
  V_{eff} & = & \frac{(p_\alpha - p_\gamma\cos\beta)^2}{2 I_1 \sin^2\beta} + \frac{p^2_\gamma}{2 I_3}.
 \end{eqnarray*}
 Draw schematically how the effective potential energy $V_{eff}$ changes with $\beta$.  What are the two constant-$\beta$ solutions where $V'_{eff}(\beta) = 0$?  Discuss any conditions you might encounter.  What is the resulting total energy corresponding to each solution?
 \item Fetter \& Walecka, Problem 5.9.
 \item Fetter \& Walecka, Problem 6.13.
\end{enumerate}

\end{document}
