\documentclass[letterpaper,11pt]{article}
\usepackage[utf8x]{inputenc}
\usepackage{enumerate}
\usepackage{enumitem}
\usepackage{fullpage}
\usepackage{amsmath}
\usepackage{amssymb}
\usepackage{mathrsfs}

\usepackage{pgf}
\usepackage{tikz}

%opening
\title{Physics 601 (Fall 2012) \\ Homework Assignment 8: Solutions}
\date{Due: Friday November 2, 2012}

\begin{document}

\maketitle

\paragraph*{Small Oscillations}
\begin{enumerate}
 \item Fetter \& Walecka, Problem 4.6.
\end{enumerate}
a) The Lagrangian in polar coordinates in the orbital plane is
\begin{equation*}
 L = \frac{1}{2} m \left(\dot{r}^2 + r^2\dot{\phi}^2\right) - V(r).
\end{equation*}
The Euler-Lagrange equation in $r$ is
\begin{eqnarray*}
 \frac{d}{dt}\frac{\partial L}{\partial \dot{r}} & = & \frac{\partial L}{\partial r} \\
 m \ddot{r} & = & m r \dot\phi^2 - V'(r).
\end{eqnarray*}
At the equilibrium, $r = r_0$ and $\phi = \Omega t$, this gives $V'(r_0) = m r_0 \Omega^2$.

With $r = r_0 + \eta_r$ and $\phi = \Omega t + \eta_\phi$, we can expand the Lagrangian:
\begin{eqnarray*}
 L & = & \frac{1}{2} m \left(\dot{\eta_r}^2 + (r_0 + \eta_r)^2 (\Omega + \dot\eta_\phi)^2\right) - V(r_0 + \eta_r) \\
   & = & \frac{1}{2} m \left(\dot{\eta_r}^2 + (r_0^2 + 2 r_0 \eta_r + \eta_r^2) (\Omega^2 + 2 \Omega \dot\eta_\phi + \dot\eta_\phi^2)\right) - V(r_0 + \eta_r) \\
   & = & \frac{1}{2} m \left(\dot{\eta_r}^2 + r_0^2 \Omega^2 + 2 r_0 \Omega^2 \eta_r + \Omega^2 \eta_r^2 + 2 r_0^2 \Omega \dot\eta_\phi + 4 r_0 \Omega \eta_r \dot\eta_\phi + 2 \Omega \dot\eta_\phi \eta_r^2 + r_0^2 \dot\eta_\phi^2  + 2 r_0 \eta_r \dot\eta_\phi^2 + \eta_r^2 \dot\eta_\phi^2 \right) \\
   & &  - V(r_0) - \eta_r V'(r_0) - \frac{1}{2} \eta_r^2 V''(r_0) + \mathcal{O}(\eta^3) \\
   & = & \frac{1}{2} m \left(\dot{\eta_r}^2 + r_0^2 \Omega^2 + \Omega^2 \eta_r^2 + 2 r_0^2 \Omega \dot\eta_\phi + 4 r_0 \Omega \eta_r \dot\eta_\phi + 2 \Omega \dot\eta_\phi \eta_r^2 + r_0^2 \dot\eta_\phi^2  + 2 r_0 \eta_r \dot\eta_\phi^2 + \eta_r^2 \dot\eta_\phi^2 \right) \\
   & & - V(r_0) - \frac{1}{2} \eta_r^2 V''(r_0) + \mathcal{O}(\eta^3),
\end{eqnarray*}
where we used the previously found expression for $V'(r_0)$.

Since the Lagrangian is cyclic in $\phi$, the angular momentum $\ell = p_\phi$ is conserved.  After expansion,
\begin{eqnarray*}
 \ell = p_\phi = \frac{\partial L}{\partial \dot{\phi}} & = & m r^2 \dot\phi \\
 & = & m (r_0 + \eta_r)^2 (\Omega + \dot\eta_\phi) \\
 & = & m r_0^2 \Omega + m r_0^2 \dot\eta_\phi + 2 m r_0 \Omega \eta_r + 2 m r_0 \eta_r \dot\eta_\phi + m \Omega \eta_r^2 + m \eta_r^2 \dot\eta_\phi
\end{eqnarray*}
the angular momentum $\ell$ is constant and equal to $m r_0^2 \Omega$.  This results in
\begin{equation*}
 r_0^2 \dot\eta_\phi + 2 r_0 \eta_r \dot\eta_\phi + \eta_r^2 \dot\eta_\phi = - 2 r_0 \Omega \eta_r - \Omega \eta_r^2.
\end{equation*}
We can use this relation (multiplied with $2 \Omega$) to remove the non-bilinear terms in the expanded Lagrangian, after removing constant terms:
\begin{eqnarray*}
 L & = & \frac{1}{2} m \left(\dot{\eta_r}^2 + \Omega^2 \eta_r^2 + 2 r_0^2 \Omega \dot\eta_\phi + 4 r_0 \Omega \eta_r \dot\eta_\phi + 2 \Omega \dot\eta_\phi \eta_r^2 + r_0^2 \dot\eta_\phi^2  + 2 r_0 \eta_r \dot\eta_\phi^2 + \eta_r^2 \dot\eta_\phi^2\right) - \frac{1}{2} \eta_r^2 V''(r_0) + \mathcal{O}(\eta^3) \\
   & = & \frac{1}{2} m \left(\dot{\eta_r}^2 + r_0^2 \dot\eta_\phi^2 - 3 \Omega^2 \eta_r^2\right) - \frac{1}{2} \eta_r^2 V''(r_0) + \mathcal{O}(\eta^3).
\end{eqnarray*}
This is in the form $L = \frac{1}{2} \dot\eta^T M \dot\eta - \frac{1}{2} \eta^T V \eta$ with
\begin{eqnarray*}
 M & = & m \left[
  \begin{array}{cc}
   1 & 0 \\
   0 & r_0^2 \\
  \end{array} \right], \\
 V & = & \left[
  \begin{array}{cc}
   V''(r_0) + 3 m \Omega^2 & 0 \\
   0 & 0 \\
  \end{array} \right].
\end{eqnarray*}

The eigenvalue equation $\det(V - \omega^2 M) = 0$ becomes then
\begin{eqnarray*}
 \left|   \begin{array}{cc}
   V''(r_0) + 3 m \Omega^2 - m \omega^2 & 0 \\
   0 & -m r_0^2 \omega^2 \\
 \end{array} \right| & = & 0, \\
 -m r_0^2 \omega^2 \left(V''(r_0) + 3 m \Omega^2 - m \omega^2 \right) & = & 0.
\end{eqnarray*}

The non-zero eigenvalue of this system is
\begin{equation*}
 \omega^2 = 3 \Omega^2 + \frac{1}{m} V''(r_0) = \Omega^2 \left(3 + r_0 \frac{V''(r_0)}{V'(r_0)} \right).
\end{equation*}
This eigenvalue is positive, and the equilibrium therefore stable, as long as the following criterion is satisfied:
\begin{equation*}
 V''(r_0) + \frac{3}{r_0} V'(r_0) > 0.
\end{equation*}

b) We can derive the same criterium from the one-dimensional Lagrangian with an effective potential $V_{eff}(r)$
\begin{equation*}
 V_{eff}(r) = V(r) + \frac{\ell^2}{2 m r^2}.
\end{equation*}
Note the difference with (3.9) where the energy $E$ is considered.  The criterion for stability at the equilibrium is now, with $\ell = m r^2 \Omega$ as found above,
\begin{eqnarray*}
 V''_{eff}(r_0) & > & 0 \\
 V''(r_0) + \frac{3 \ell^2}{m r_0^4} & > & 0 \\
 V''(r_0) + 3 m \Omega^2 & > & 0 \\
 V''(r_0) + \frac{3}{r_0} V'(r_0) & > & 0,
\end{eqnarray*}
where we used again the expression for $V'(r_0)$.

c) If $V(r) = -\lambda r^{-n}$, then the criterion becomes
\begin{equation*}
 - n (n+1) \lambda r_0^{-n-2} + 3 n \lambda r_0^{-n-2} > 0.
\end{equation*}
This is satisfied for $n(n+1) - 3 n < 0$ or $n+1 < 3$ or $n < 2$.

d) If $V(r) = -\frac{\lambda}{r} \exp(-\frac{r}{a})$, then the criterion becomes
\begin{eqnarray*}
 \left( -\frac{1}{a^2} \frac{\lambda}{r_0} \exp(-\frac{r_0}{a}) - \frac{1}{a} \frac{\lambda}{r_0^2} \exp(-\frac{r_0}{a}) - 2 \frac{\lambda}{r_0^3} \exp(-\frac{r_0}{a}) - \frac{1}{a} \frac{\lambda}{r_0^2} \exp(-\frac{r_0}{a}) \right) & & \\
+ 3 \frac{1}{r_0} \left( \frac{1}{a} \frac{\lambda}{r_0} \exp(-\frac{r_0}{a}) + \frac{\lambda}{r_0^2} \exp(-\frac{r_0}{a}) \right) & > & 0 \\
 \left( -\frac{1}{a^2} - \frac{1}{a r_0} - 2 \frac{1}{r_0^2} - \frac{1}{a r_0} + 3 \frac{1}{a r_0} + 3 \frac{1}{r_0^2} \right) \frac{\lambda}{r_0} \exp(-\frac{r_0}{a}) & > & 0 \\
 \left( \frac{1}{r_0^2} + \frac{1}{a r_0} - \frac{1}{a^2}\right) & > & 0 \\
 1 + \frac{r_0}{a} - \left(\frac{r_0}{a}\right)^2 & > & 0 \\
 \left( \frac{\sqrt{5} - 1}{2} + \frac{r_0}{a} \right) \left( \frac{\sqrt{5} + 1}{2} - \frac{r_0}{a} \right) & > & 0.
\end{eqnarray*}
This is satisfied if $r_0/a$ is smaller than the golden ratio, or
\begin{equation*}
 r_0 < \frac{\sqrt{5} + 1}{2} \, a.
\end{equation*}

\begin{enumerate}[resume]
 \item Fetter \& Walecka, Problem 4.7.
\end{enumerate}
a) The potential energy is now $V(r) = V_0(r) + \delta V(r)$.  The angle that the orbit describes for a time $\Delta t$ is $\Delta\phi = \omega \Delta t$.  For the unperturbed circular orbit, the angular frequency is $\Omega$, and $\Omega T = 2\pi$.  For the nearly circular orbit in this problem, the angular frequency changes to $\omega$ as in the previous problem.  The additional angle due to the perturbation is therefore
\begin{eqnarray*}
 \delta\phi & = & \Delta\phi^{\hbox{(pert)}} - \Delta\phi^{\hbox{(unpert)}} \\
 & = & \Omega T \left(3 + r_0 \frac{V''(r_0)}{V'(r_0)}\right)^{1/2} - \Omega T \left(3 + r_0 \frac{V_0''(r_0)}{V_0'(r_0)}\right)^{1/2} \\
 & = & 2\pi \left(3 + r_0 \frac{V''(r_0)}{V'(r_0)}\right)^{1/2} - 2\pi \left(3 + r_0 \frac{V_0''(r_0)}{V_0'(r_0)}\right)^{1/2}.
\end{eqnarray*}
We now expand this for small $\delta V(r)$ as
\begin{eqnarray*}
 \delta\phi & = & 2\pi \left(3 + \frac{1}{2} r_0 \frac{V''(r_0)}{V'(r_0)}\right) - 2\pi \left(3 + \frac{1}{2} r_0 \frac{V_0''(r_0)}{V_0'(r_0)}\right) \\
 & = & \pi r_0 \frac{1}{V_0'(r_0)} \left(V_0''(r_0) + \delta V''(r_0)\right) \left(1 - \frac{\delta V'(r_0)}{V_0'(r_0)}\right) - \pi r_0 \frac{1}{V_0'(r_0)} V_0''(r_0) \\
 & = & \pi r_0 \frac{1}{V_0'(r_0)} \left( - \frac{V_0''(r_0)}{V_0'(r_0)} \delta V'(r_0) + \delta V''(r_0)\right) + \mathcal{O}(\delta V^2).
\end{eqnarray*}
Using the expression for $V_0(r)$ we find
\begin{eqnarray*}
 V_0(r) & = & -mMGr^{-1}, \\
 V_0'(r) & = & mMGr^{-2}, \\
 V_0''(r) & = & -2mMGr^{-3}, \\
 \frac{V_0''(r)}{V_0'(r)} & = & -2 r^{-1}.
\end{eqnarray*}
This finally leads to
\begin{eqnarray*}
 \delta\phi & = & \pi r_0 \frac{r_0^2}{mMG} \left( 2 r_0^{-1} \delta V'(r_0) + \delta V''(r_0)\right) + \mathcal{O}(\delta V^2) \\
 & = & \frac{\pi r_0}{mMG} \left( 2 r_0 \delta V'(r_0) + r_0^2 \delta V''(r_0)\right) + \mathcal{O}(\delta V^2).
\end{eqnarray*}

b) For $\delta V(r) = -\alpha m(MG/rc)^2$ this leads to
\begin{eqnarray*}
 \delta V'(r) & = & 2 \alpha m (MG/c)^2 r^{-3}, \\
 \delta V''(r) & = & - 6 \alpha m (MG/c)^2 r^{-4},
\end{eqnarray*}
and we find for the precession
\begin{equation*}
 \delta\phi = -2 \alpha \frac{\pi}{r_0} \frac{MG}{c^2}.
\end{equation*}

\begin{enumerate}[resume]
 \item Fetter \& Walecka, Problem 4.9, (a).
\end{enumerate}
The molecules have positions $(\eta_1,\eta_2)$, $(a+\eta_3,\eta_4)$ and $(\eta_5,a+\eta_6)$, with $a$ the equilibrium distance between the atom at the right angle and each of its neighbors.  We call the two atoms on the $x$ axis the $x$ pair, similarly for the $y$ pair, and the remaining pair is the diagonal or $xy$ pair.  The potential energy of the $x$ pair is
\begin{eqnarray*}
 V_x & = & \frac{1}{2} k \left( \sqrt{(a + \eta_3 - \eta_1)^2 + (\eta_4 - \eta_2)^2}  - a\right)^2 \\
   & = & \frac{1}{2} k \left( \sqrt{a^2 + \eta_3^2 + \eta_1^2 - 2\eta_1\eta_3 + 2 a \eta_3 - 2 a \eta_1 + \eta_4^2 + \eta_2^2 - 2 \eta_2 \eta_4} - a \right)^2 \\
   & = & \frac{1}{2} k \left( a + \frac{1}{2a} \left( \eta_3^2 + \eta_1^2 - 2 \eta_1 \eta_3 + 2 a \eta_3 - 2 a \eta_1 + \eta_4^2 + \eta_2^2 - 2 \eta_2 \eta_4 \right) - a\right)^2 \\
   & = & \frac{1}{4a} k \left( \eta_3^2 + \eta_1^2 - 2 \eta_1 \eta_3 + 2 a \eta_3 - 2 a \eta_1 + \eta_4^2 + \eta_2^2 - 2 \eta_2 \eta_4 \right)^2 \\
   & = & \frac{1}{4a} k \left( 2 a (\eta_3 - \eta_1) \right)^2 + \mathcal{O}(\eta^3) \\
   & = & \frac{1}{2} k (\eta_3 - \eta_1)^2.
\end{eqnarray*}
This indicates that, indeed, we can ignore the transverse displacements to the second order required for this problem.  For the potential energy of the $y$ pair we find now immediately
\begin{equation*}
 V_y = \frac{1}{2} k (\eta_6 - \eta_2)^2.
\end{equation*}
The potential energy of the diagonal pair will also depend to second order only on the linear distance (itself to first order):
\begin{equation*}
 V_{xy} = \frac{1}{2} k \left( \frac{(\eta_4 - \eta_3)}{\sqrt{2}} - \frac{(\eta_6 - \eta_5)}{\sqrt{2}} \right).
\end{equation*}
When we combine the full potential energy, we find
\begin{eqnarray*}
 V & = & V_x + V_y + V_{xy} \\
 & = & \frac{1}{2} k (\eta_3 - \eta_1)^2 + \frac{1}{2} k (\eta_6 - \eta_2)^2 + \frac{1}{2} k \left( \frac{(\eta_4 - \eta_3)}{\sqrt{2}} - \frac{(\eta_6 - \eta_5)}{\sqrt{2}} \right) \\
 & = & \frac{1}{2} k (\eta_3 - \eta_1)^2 + \frac{1}{2} k (\eta_6 - \eta_2)^2 + \frac{1}{4} k (\eta_4 - \eta_3)^2 + \frac{1}{4} k (\eta_6 - \eta_5)^2 - \frac{1}{2} k (\eta_4 - \eta_3) (\eta_6 - \eta_5).
\end{eqnarray*}
The mass and potential matrices are
\begin{eqnarray*}
 M & = & m\,1, \\
 V & = & \frac{1}{2} k \left[ \begin{array}{cccccc}
 2 & 0 & -2 & 0 & 0 & 0 \\
 0 & 2 & 0 & 0 & 0 & -2 \\
 -2 & 0 & 3 & -1 & -1 & 1 \\
 0 & 0 & -1 & 1 & 1 & -1 \\
 0 & 0 & -1 & 1 & 1 & -1 \\
 0 & -2 & 1 & -1 & -1 & 3
 \end{array} \right].
\end{eqnarray*}
The eigenvalue equation $\det(V - M\omega^2) = 0$ or $\det(\frac{2}{k} V - 2 \lambda^2\,1) = 0$, with $\lambda^2 =  \frac{m}{k} \omega^2$ can be quickly reduced to lower dimensional determinants using pivots on the first and second row (if one chooses not to do this with a computer algebra program):
\begin{eqnarray*}
 \det(\frac{2}{k} V - 2 \lambda^2\,1) & = & \left| \begin{array}{cccccc}
 2 - 2 \lambda^2 & 0 & -2 & 0 & 0 & 0 \\
 0 & 2 - 2 \lambda^2 & 0 & 0 & 0 & -2 \\
 -2 & 0 & 3 - 2 \lambda^2 & -1 & -1 & 1 \\
 0 & 0 & -1 & 1 - 2 \lambda^2 & 1 & -1 \\
 0 & 0 & -1 & 1 & 1 - 2 \lambda^2 & -1 \\
 0 & -2 & 1 & -1 & -1 & 3 - 2 \lambda^2
 \end{array} \right| \\
 & = & 2 \left( 1 - \lambda^2 \right) \left| \begin{array}{ccccc}
 2 - 2 \lambda^2 & 0 & 0 & 0 & -2 \\
 0 & 3 - 2 \lambda^2 & -1 & -1 & 1 \\
 0 & -1 & 1 - 2 \lambda^2 & 1 & -1 \\
 0 & -1 & 1 & 1 - 2 \lambda^2 & -1 \\
 -2 & 1 & -1 & -1 & 3 - 2 \lambda^2
 \end{array} \right| \\
 & & - 2 \left| \begin{array}{ccccc}
 0 & -2 & 0 & 0 & 0 \\
 2 - 2 \lambda^2 & 0 & 0 & 0 & -2 \\
 0 & -1 & 1 - 2 \lambda^2 & 1 & -1 \\
 0 & -1 & 1 & 1 - 2 \lambda^2 & -1 \\
 -2 & 1 & -1 & -1 & 3 - 2 \lambda^2
 \end{array} \right| \\
 & = & 4 \left( 1 - \lambda^2 \right)^2 \left| \begin{array}{cccc}
 3 - 2 \lambda^2 & -1 & -1 & 1 \\
 -1 & 1 - 2 \lambda^2 & 1 & -1 \\
 -1 & 1 & 1 - 2 \lambda^2 & -1 \\
 1 & -1 & -1 & 3 - 2 \lambda^2
 \end{array} \right| \\
 & & - 4 \left( 1 - \lambda^2 \right) \left| \begin{array}{cccc}
 0 & 3 - 2 \lambda^2 & -1 & -1 \\
 0 & -1 & 1 - 2 \lambda^2 & 1 \\
 0 & -1 & 1 & 1 - 2 \lambda^2 \\
 -2 & 1 & -1 & -1
 \end{array} \right| \\
 & & - 4 \left| \begin{array}{cccc}
 2 - 2 \lambda^2 & 0 & 0 & -2 \\
 0 & 1 - 2 \lambda^2 & 1 & -1 \\
 0 & 1 & 1 - 2 \lambda^2 & -1 \\
 -2 & -1 & -1 & 3 - 2 \lambda^2
 \end{array} \right|
\end{eqnarray*}
\begin{eqnarray*}
 & = & 4 \left( 1 - \lambda^2 \right)^2 \left| \begin{array}{cccc}
 3 - 2 \lambda^2 & -1 & -1 & 1 \\
 -1 & 1 - 2 \lambda^2 & 1 & -1 \\
 -1 & 1 & 1 - 2 \lambda^2 & -1 \\
 1 & -1 & -1 & 3 - 2 \lambda^2
 \end{array} \right| \\
 & & - 8 \left( 1 - \lambda^2 \right) \left| \begin{array}{cccc}
 3 - 2 \lambda^2 & -1 & -1 \\
 -1 & 1 - 2 \lambda^2 & 1 \\
 -1 & 1 & 1 - 2 \lambda^2 \\
 \end{array} \right| \\
 & & - 8 \left( 1 - \lambda^2 \right) \left| \begin{array}{ccc}
 1 - 2 \lambda^2 & 1 & -1 \\
 1 & 1 - 2 \lambda^2 & -1 \\
 -1 & -1 & 3 - 2 \lambda^2
 \end{array} \right| \\
 & & - 8 \left| \begin{array}{ccc}
 0 & 0 & -2 \\
 1 - 2 \lambda^2 & 1 & -1 \\
 1 & 1 - 2 \lambda^2 & -1 \\
 \end{array} \right| \\
 & = & 4 \left( 1 - \lambda^2 \right)^2 \left| \begin{array}{cccc}
 3 - 2 \lambda^2 & -1 & -1 & 1 \\
 -1 & 1 - 2 \lambda^2 & 1 & -1 \\
 -1 & 1 & 1 - 2 \lambda^2 & -1 \\
 1 & -1 & -1 & 3 - 2 \lambda^2
 \end{array} \right| \\
 & & - 8 \left( 1 - \lambda^2 \right) \left| \begin{array}{cccc}
 3 - 2 \lambda^2 & -1 & -1 \\
 -1 & 1 - 2 \lambda^2 & 1 \\
 -1 & 1 & 1 - 2 \lambda^2 \\
 \end{array} \right| \\
 & & - 8 \left( 1 - \lambda^2 \right) \left| \begin{array}{ccc}
 1 - 2 \lambda^2 & 1 & -1 \\
 1 & 1 - 2 \lambda^2 & -1 \\
 -1 & -1 & 3 - 2 \lambda^2
 \end{array} \right| \\
 & & + 16 \left| \begin{array}{ccc}
 1 - 2 \lambda^2 & 1 \\
 1 & 1 - 2 \lambda^2 \\
 \end{array} \right| \\
 & = & 4 \left( 1 - \lambda^2 \right)^2 \left| \begin{array}{cccc}
 3 - 2 \lambda^2 & -1 & -1 & 1 \\
 -1 & 1 - 2 \lambda^2 & 1 & -1 \\
 -1 & 1 & 1 - 2 \lambda^2 & -1 \\
 1 & -1 & -1 & 3 - 2 \lambda^2
 \end{array} \right| \\
 & & - 16 \left( 1 - \lambda^2 \right) \left| \begin{array}{cccc}
 3 - 2 \lambda^2 & -1 & -1 \\
 -1 & 1 - 2 \lambda^2 & 1 \\
 -1 & 1 & 1 - 2 \lambda^2 \\
 \end{array} \right| \\
 & & + 16 \left( (1 - 2 \lambda^2)^2 - 1 \right) \\
 & = & 4 \left( 1 - \lambda^2 \right)^2 \left| \begin{array}{cccc}
 3 - 2 \lambda^2 & -1 & -1 & 1 \\
 -1 & 1 - 2 \lambda^2 & 1 & -1 \\
 -1 & 1 & 1 - 2 \lambda^2 & -1 \\
 1 & -1 & -1 & 3 - 2 \lambda^2
 \end{array} \right| \\
 & & - 16 \left( 1 - \lambda^2 \right) \left| \begin{array}{cccc}
 3 - 2 \lambda^2 & -1 & -1 \\
 -1 & 1 - 2 \lambda^2 & 1 \\
 -1 & 1 & 1 - 2 \lambda^2 \\
 \end{array} \right| \\
 & & - 64 \lambda^2 (1 - \lambda^2).
\end{eqnarray*}
We can factor out $(1 - \lambda^2)$ and take linear combinations of rows to simplify:
\begin{eqnarray*}
 \left| \begin{array}{cccc}
 3 - 2 \lambda^2 & -1 & -1 & 1 \\
 -1 & 1 - 2 \lambda^2 & 1 & -1 \\
 -1 & 1 & 1 - 2 \lambda^2 & -1 \\
 1 & -1 & -1 & 3 - 2 \lambda^2
 \end{array} \right| & = & \left| \begin{array}{cccc}
 3 - 2 \lambda^2 & -1 & -1 & 1 \\
 -1 & 1 - 2 \lambda^2 & 1 & -1 \\
 0 & 2 \lambda^2 & - 2 \lambda^2 & 0 \\
 0 & - 2 \lambda^2 & 0 & 2 - 2 \lambda^2
 \end{array} \right| \\
 & = & \left| \begin{array}{cccc}
 3 - 2 \lambda^2 & -1 & 0 & 0 \\
 -1 & 1 - 2 \lambda^2 & 2 \lambda^2 & - 2 \lambda^2 \\
 0 & 2 \lambda^2 & 4 \lambda^2 & 2 \lambda^2 \\
 0 & - 2 \lambda^2 & 2 \lambda^2 & 2 - 4 \lambda^2
 \end{array} \right| \\
 & = & (3 - 2 \lambda^2) \left| \begin{array}{ccc}
 1 - 2 \lambda^2 & 2 \lambda^2 & - 2 \lambda^2 \\
 2 \lambda^2 & 4 \lambda^2 & 2 \lambda^2 \\
 - 2 \lambda^2 & 2 \lambda^2 & 2 - 4 \lambda^2
 \end{array} \right| \\
 & & + \left| \begin{array}{ccc}
 -1 & 0 & 0 \\
 2 \lambda^2 & 4 \lambda^2 & 2 \lambda^2 \\
 - 2 \lambda^2 & 2 \lambda^2 & 2 - 4 \lambda^2
 \end{array} \right| \\
 & = & - 16 \lambda^2 (1 - \lambda^2) (\lambda^4 - 3 \lambda^2 + 1) \\
\left| \begin{array}{cccc}
 3 - 2 \lambda^2 & -1 & -1 \\
 -1 & 1 - 2 \lambda^2 & 1 \\
 -1 & 1 & 1 - 2 \lambda^2 \\
 \end{array} \right| & = & \left| \begin{array}{cccc}
 3 - 2 \lambda^2 & -1 & -1 \\
 -1 & 1 - 2 \lambda^2 & 1 \\
 0 & 2 \lambda^2 & - 2 \lambda^2 \\
 \end{array} \right| \\
 & = & \left| \begin{array}{cccc}
 3 - 2 \lambda^2 & -1 & 0 \\
 -1 & 1 - 2 \lambda^2 & 2 \lambda^2 \\
 0 & 2 \lambda^2 & - 4 \lambda^2
 \end{array} \right| \\
 & = & (3 - 2 \lambda^2) \left[ -4 \lambda^2 (1 - 2 \lambda^2) - 4 \lambda^4 \right] + 4 \lambda^2 \\
 & = & (3 - 2 \lambda^2) 4 \lambda^2 (1 + \lambda^2) + 4 \lambda^2 \\
 & = & 4 \lambda^2 (2 - \lambda^2) (1 - 2 \lambda^2).
\end{eqnarray*}
After all this tedious algebra we finally find:
\begin{equation*}
 \det(\frac{2}{k} V - 2 \lambda^2\,1) = \lambda^6 (1 - \lambda^2) (2 - \lambda^2) (3 - \lambda^2).
\end{equation*}
There are three zero frequency modes, two for uniform translational motion and one for uniform rotational motion around the center of mass.  The non-zero frequencies are $\omega_1^2 = \frac{k}{m}$, $\omega_2^2 = 2 \frac{k}{m}$, and $\omega_3^2 = 3 \frac{k}{m}$.


\end{document}
