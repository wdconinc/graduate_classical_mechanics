\documentclass[letterpaper,11pt]{article}
\usepackage[utf8x]{inputenc}
\usepackage{enumerate}
\usepackage{enumitem}
\usepackage{fullpage}
\usepackage{amsmath}

\usepackage{pgf}
\usepackage{tikz}
\usetikzlibrary{arrows,shapes,trees}

%opening
\title{Midterm Exam \\ Classical Mechanics \\ Physics 601 (Fall 2011)}
\date{Due by noon on Wednesday October 12, 2011}

\begin{document}

\maketitle

\paragraph*{Instructions}
\begin{itemize}
 \item This midterm exam is to be completed \textbf{individually}.
 \item Please carefully explain each step in your answer.
\end{itemize}

\paragraph*{Lagrangian Mechanics}
\begin{enumerate}
 \item Two masses $m_1$ and $m_2$ are connected by a string with fixed length passing through a hole in a smooth table, so that $m_1$ rests on the table and $m_2$ hangs suspended and moves only in the vertical direction.  Consider the motion until $m_1$ reaches the hole.
 \begin{itemize}
  \item Write down the Lagrangian and the equations of motion for this system. Interpret the first integral you encounter.
  \item Under what condition will the hanging mass remain stationary?
  \item Starting from this stationary situation, the mass $m_2$ is pulled down slightly and released again. Investigate the resulting small oscillations in the vertical position of $m_2$.
  \item Calculate the tension in the string for stationary motion using the method of the Lagrange multipliers.
 \end{itemize}
 \item A particle of mass $m$ in a uniform gravitational field is constrained to move on a helical wire with height $z = a\theta$ and radius $r = b$ in cylindrical coordinates.
 \begin{itemize}
  \item Write down the Lagrangian and the equations of motion for this system.
  \item Solve the equations of motion if at $t = 0$ the mass is at $z = 0$.
  \item Find the forces of constraint using the method of the Lagrange multipliers.
 \end{itemize}
\end{enumerate}

\paragraph*{Hamiltonian Mechanics}
\begin{enumerate}[resume]
 \item The point of suspension of a simple pendulum of length $\ell$ and mass $m$ is constrained to move on the parabola $z = a x^2$ in a vertical plane.  Derive the Hamiltonian and the equations of motion for this system.  \textit{Note:} You can write the kinetic energy as the quadratic form $T = \frac{1}{2} \dot{q}^T M \dot{q} = \frac{1}{2} p^T M^{-1} p$ for some symmetric matrix $M$.
\end{enumerate}

\paragraph*{Hamiltonian-Jacobi Theory}
\begin{enumerate}[resume]
 \item Consider the physical system described by the kinetic energy $T$ and potential energy $V$
 \begin{eqnarray*}
  T & = & \frac{1}{2} (\dot{q}_1^2 + \dot{q}_2^2) (q_1^2 + q_2^2), \\
  V & = & (q_1^2 + q_2^2)^{-1},
 \end{eqnarray*}
 where $q_1$ and $q_2$ are the generalized coordinates.  What is the Hamilton-Jacobi equation for this system?  Solve this equation to find Hamilton's principal function, and determine the solutions to the equations of motion (you need not evaluate any definite integrals).
\end{enumerate}


\end{document}
